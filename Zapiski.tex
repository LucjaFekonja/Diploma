\documentclass[mat1]{fmfdelo}
\usepackage{amsmath}
\usepackage{graphicx}

\avtor{Lucija Fekonja}
\naslov{Minimalne ploskve in Björlingov problem}
\title{Minimal surfaces and Björling problem}

\mentor{Doc. dr. Uroš Kuzman}

\letnica{2022}

\povzetek{}
\abstract{}


\klasifikacija{navedite vsaj eno klasifikacijsko oznako -- dostopne so na \url{www.ams.org/mathscinet/msc/msc2020.html}}
\kljucnebesede{navedite nekaj ključnih pojmov, ki nastopajo v delu} % navedite nekaj ključnih pojmov, ki nastopajo v delu
\keywords{angleški prevod ključnih besed} % angleški prevod ključnih besed

\zapisiMetaPodatke  % poskrbi za metapodatke in veljaven PDF/A-1b standard

\newcommand{\R}{\mathbb R}
\newcommand{\N}{\mathbb N}
\newcommand{\Z}{\mathbb Z}
\newcommand{\C}{\mathbb C}
\newcommand{\Q}{\mathbb Q}

% Še neke ostale stvari ...

\begin{document}

\section{Uvod}

Matematika opisuje svet okoli nas. Še posebej lepo in nadzorno lahko z njo opišemo razne oblike, ki jih 
opazujemo v naravi. Tako lahko parametriziramo mnogo različnih ploskev - od ravnine do najzanimivejših oblik.
S preslikavami lahko navedemo, kakšna bo oblika milnega filma, ki nastane, ko neko prazno telo potopimo v milnico.
Ko je milni mehurček v ravnovesuju, to se zgodi, ko je njegova vizkoznost - odpor tekočine na spremembo svoje 
oblike - enaka nič, je njegova oblika takšna, da ima kar se da malo površino. Matematično temu rečemo, da je milni 
mehurček minimalna ploskev. V nadaljevanju bomo z uporabo idej geometrije ploskev, diferencialne geometrije,
kompleksne analize in variacijskega računa spoznali ideje, ki se tičejo minimalnih ploskev, nato pa bomo na podlagi
teh spoznanj navedli Björlingov problem ter ga dokazali.

V 18. stoletju, ko so se minimalne ploskve komaj začeli raziskovati, so jih matematiki poznali le malo. Kot eno od teh 
minimalnih ploskev so opredelili ravnino, leta $1741$ je Leonhard Euler odkril katenoido, še kasneje je Jean Babtiste Meusnier
odkril helokoid leta $1776$. Naslednji je novi minimalni ploskvi odkril Scherk leta $1835$. 
V 19. stoletju so minimalne ploskve dobile veliko pozornosti in s tem novih naukov zahvaljujoč 
matematikom kot so Riemann, Scherk, Schwarz, Enneper in Weierstrass. S pomočjo takrat na novo odkrite kompleksne analize so 
ploskve analitično opisali. V tem času je deloval tudi fizik in matematik Plateau, ki je raziskoval površinsko napetost,
s čimer se je izkazalo, da imajo minimalne ploskve precejšen fizikalni pomen, in niso le abstraktni pojem. Raziskovanje 
minimalnih ploskev se je nadaljevalo v 20. stoletje, ko se je govorilo o diferencialnih enačbah, konformnih funkcijah, 
Riemannovih ploskvah, funkcionalni analizi in mnogem drugem. Iznajdba računalnika je pripomogla k odkrivanju novih minimalnih
ploskev, s čemer pa so nastajali tudi novi problemi in razni načini klasifikacij. Raziskovanje minimalnih ploskev še zdaleč ni 
končano. Ostajajo še nerešeni problemi in odprta vprašanja, ki čakajo, da bodo odgovorjena. 


\pagebreak

\section{Definicija minimalne ploskve}

Minimalno ploskev lahko definiramo na $8$ različnih, a med sabo ekvavilentnih načinov.
Intuitivno je minimalna ploskev takšna ploskev, ki ima najmanjšo površino pri nekih robnih pogojih. 
Lokalno lahko definicijo minimalne ploskve zapišemo na sledeč način.

\begin{definicija}
    Ploskev $M \subset \mathbb{R} ^3$ je \emph{minimalna ploskev}, če in samo če obstaja okolica vsake točke $p \in M$, omejena s 
    sklenjeno krivuljo, ki ima najmanjšo površino med vsemi ploskvami z isto omejitvijo.
\end{definicija}


Pogosteje jih definiramo z uporabo srednje ukrivljenosti. Na ta način jih podamo  globalno.
Zavoljo te definicije spoznajmo nekaj osnovnih pojmov vezanih na ploskve v splošnem.

\begin{definicija}
    \emph{Vektorja hitrosti} ploskve parametrizirane z 
    \begin{align*}
        \phi : & D \to \R^3 \\
        & (u, v) \mapsto (x(u, v), y(u, v), z(u, v))
    \end{align*}
    sta definirana kot 
    \begin{align*}
        \phi_u = \frac{d\phi}{du} = (\frac{dx}{du}, \frac{dy}{du}, \frac{dz}{du})  \\
        \phi_v = \frac{d\phi}{dv} = (\frac{dx}{dv}, \frac{dy}{dv}, \frac{dz}{dv})
    \end{align*}
\end{definicija}

\begin{definicija}
    Ploskev $S$ s parametrizacijo $\phi = (x, y, z)$ je \emph{regularna}, če za njo velja:
    \begin{itemize}
        \item parametrizacija $\phi$ je gladka, kar pomeni, da imajo funkcije $x, y \text{ in } z$ zvezne parcialne odvode vseh redov,
        \item $\phi$ je homeomorfizem, 
        \item regularnostni pogoj, ki pravi, da v nobeni točki $p \in S$ vektorski produkt vektorjev hitrosti ni enak nič: $\phi_u \times \phi_v \neq 0$. To pomeni, da sta vektorja hitrosti linearno neodvisna.
    \end{itemize}    
\end{definicija}

\begin{definicija}
    \emph{Tangentni prostor $T_p S$} je prostor, ki ga raztezata vektorja hitrosti v točki $p$.
    $$ T_p S = Lin\{ \phi_u (p), \phi_v (p) \} $$
\end{definicija}

Vsi tangentni vektorji na ploskev v točki $p \in S$ ležijo na tangentni ravnini, ki jo raztezata linearno neodvisna vektorja hitrosti.

\begin{definicija}
    Naj bo $\alpha : I \to \R^3$ krivulja v evklidskem prostoru, kjer je $I \subset \R$ interval.
    \emph{Ploskovna krivulja $\gamma$} je projekcija krivulje $\alpha$ glede na projekcijo $\phi$. Dana je s predpisom:
    \begin{align*}
        \gamma : & \left[ 0, \epsilon \right] \to S \\
        & s \mapsto \phi(\alpha(s))
    \end{align*}
\end{definicija}

\begin{definicija}
    \emph{Ukrivljenost $K$} krivulje $\gamma$ je difinirana kot $K = \left\lvert \frac{dT}{ds} \right\rvert $, kjer je $T$ enotski tangentni vektor dane krivulje $T = \frac{\gamma'(s)}{\left\lvert \gamma'(s) \right\rvert }$
\end{definicija}

\begin{definicija}
    \emph{Enotsko normalo $N$} regularne ploskve s parametrizacijo $\phi(u, v)$ definiramo kot vektor $N = \frac{\phi_u \times \phi_v}{\left\lvert \phi_u \times \phi_v \right\rvert }$.
\end{definicija}

\begin{definicija}
    \emph{Stranska normala $B$} v točki $p \in S$ je vektor $B = N \times T$, kjer je $N$ enotska normala in $T$ tangentni vektor iz $T_p S$.
\end{definicija}

\begin{definicija}
    \emph{Normalna ukrivljenost $K_n$} je komponenta ukrivljenosti $K$ ploskovne krivulje $\gamma$ v smeri normale.
    $$ K_n = \frac{dT}{dS} \cdot N $$
\end{definicija}

\begin{definicija}
    \emph{Geodestična ukrivljenost $K_g$} je komponenta ukrivljenosti $K$ ploskovne krivulje $\gamma$ v smeri stranske normale.
    $$ K_n = \frac{dT}{dS} \cdot B $$
\end{definicija}

\begin{lema}
    Ukrivljenost $K$ lahko izrazimo kot $K = \sqrt{K_n^2 + K_g^2}$
\end{lema}

\begin{definicija}
    \emph{Srednja ukrivljenost H} je definirana kot $H = \frac{ K_1 + K_2 }{2}$, kjer sta $K_1$ in $K_2$ normalni ukrivljenosti dveh pravokotnih si tangent.
\end{definicija}

Sedaj lahko globalno definiramo minimalno ploskev.

\begin{definicija}
    Ploskev se imenuje \emph{minimalna ploskev}, če je njena srednja ukrivljenost enaka nič.
\end{definicija}


\section{Prva in druga fundamentalna forma}

V tem poglavju bomo navedli še eno ekvavilentno definicijo minimalne ploskve.

\begin{definicija}
    \emph{Prva fundamentalna forma $I(u, v)$} za $u, v \in T_p S$ je simetrični operator $$I(u, v) = u \cdot v$$
\end{definicija}

Kvadratna forma prve fundamentalne forme je $I(v) = I(v, v) = v \cdot v = \left\lvert v \right\rvert^2$ 

Naj bo zdaj $\gamma$ krivulja z enotsko hitrostjo, to pomeni, da je $\left\lvert \gamma' \right\rvert = 1$ povsod na njenem definicijskem območju.
Potem je $\gamma' \cdot \gamma' = 1$.
Navedeno lahko razpišemo:

\begin{align*}
    1 &= \gamma' \cdot \gamma' \\
    &= (u' \phi_u + v' \phi_v) \cdot (u' \phi_u + v' \phi_v)\\
    &= (\phi_u \cdot \phi_u) u'^2 + (\phi_u \cdot \phi_v + \phi_v \cdot \phi_u) u' v' + (\phi_v \cdot \phi_v) v'^2  \\
    &= I(\phi_u) u'^2 + 2 I(\phi_u, \phi_v) u' v' + I(\phi_v) v'^2\\
    &= E u'^2 + 2F u' v' + G v'^2\\
\end{align*}

Označili smo:
\begin{align*}
    E &= I(\phi_u) = \phi_u \cdot \phi_u = \left\lvert \phi_u \right\rvert^2 \\
    F &= I(\phi_u, \phi_v) = \phi_u \cdot \phi_v \\
    G &= I(\phi_v) = \phi_v \cdot \phi_v = \left\lvert \phi_v \right\rvert^2
\end{align*}

Koeficienti $E, F \text{ in } G$ se imenujejo \emph{koeficienti prve fundamentalne forme $I$}.

\begin{definicija}
    \emph{Weingartenova preslikava $\Omega_p: T_p S \to T_p S$} je sebi-adjungirana linearna preslikava s predpisom
    $$ \Omega_p (v_1 \phi_u (p) + v_2 \phi_v (p)) = - v_1 N_u (p) - v_2 N_v (p) $$,
    kjer je $\phi : D \to S$ parametrizacija ploskve, $p \in D$ in $N$ enotski normalni vektor.
\end{definicija}

\begin{opomba}
    Sebi-adjungirano pomeni: $$ \forall u, v \in T_p S. \Omega_p (u) \cdot v = u \cdot \Omega_p (v) $$
\end{opomba}

\begin{definicija}
    \emph{Druga fundamentalna forma $II(u, v)$} za $u, v \in T_p S$ je simetrični operator $$II(u, v) = \Omega_p(u) \cdot v$$
\end{definicija}

\begin{lema}
    Koeficiente druge fundamentalne forme lahko izračunamo kot:
    \begin{align*}
        l &= \phi_{uu} \cdot N \\
        m &= \phi_{uv} \cdot N \\
        n &= \phi_{vv} \cdot N \\
    \end{align*}
\end{lema}

\begin{dokaz}
    Normalni vektor $N$ je pravokoten na tangentna vektorja $\phi_u$ in $\phi_v$.
    To pomeni, da sta skalarna produkta $\phi_u \cdot N = 0$ in $\phi_v \cdot N = 0$.
    Ko odvajajmo enačbi po $u$ in $v$ dobimo:
    \begin{equation*}
        \begin{aligned}[c]
            \frac{d}{du}(\phi_u \cdot N) &= 0 \\
            \frac{d}{du}(\phi_v \cdot N) &= 0 \\
        \end{aligned}
        \qquad
        \begin{aligned}[c]
            \frac{d}{dv}(\phi_v \cdot N) &= 0 \\
            \frac{d}{dv}(\phi_u \cdot N) &= 0 \\    
        \end{aligned}
    \end{equation*}
    Iz tega sledijo naslednje enakosti:
    \begin{align*}
        \phi_{uu} \cdot N &= - \phi_u \cdot N_u = l \\
        \phi_{vv} \cdot N &= - \phi_v \cdot N_u = n \\
        \phi_{uv} \cdot N &= - \phi_v \cdot N_u  \\
        \phi_{vu} \cdot N &= - \phi_u \cdot N_u  \\
    \end{align*}
    Seštejmo zadnji dve enačbi:
    $$ (\phi_{uv} + \phi_{vu}) \cdot N = - (\phi_v \cdot N_u + \phi_u \cdot N_v) = 2m $$
\end{dokaz}


Srednjo ukrivljenosti lahko sedaj podamo še nekoliko drugače:
$$ H = \frac{En + Gl - 2Fm}{2(EG - F^2)} $$,
kjer so $E, F, G, l, m \text{ in } n$ kot v zgoraj izpeljanem.

To nam zagotavlja nov kriterij preverjanja ali je ploskev minimalna. Na minimalni ploskvi mora namreč veljati $$En + Gl - 2Fm = 0$$


\section{Primeri minimalnih ploskev}

V tem poglavju bomo opisali nekaj primerov minimalnih ploskev in s pomočjo rezultatov iz prejšnjih poglavij pokazali, da 
so res minimalne.

\begin{primer}[Katenoida]
    Predstavljajmo si verigo vpeto med dva pola. Parametrično jo lahko podamo kot:
    \begin{align*}
        x(t) &= t \\
        y(t) &= \frac{1}{2}a\left( e^{\frac{t}{a}} + e^{-\frac{t}{a}} \right) \\
        &= a \cdot cosh\left( \frac{t}{a} \right) 
    \end{align*}
    kjer je $(0, a)$ točka, ki je najbližje abscizni osi.

    \begin{figure}[h]
        \includegraphics[height = 6cm]{veriznica.eps}
        \caption{Verižnica za $a \in \left\{ 0.5, 0.6, 0.7, 0.8, 0.9, 1\right\}$ }
    \end{figure}

    Če verižnico zarotiramo okoli abscizne osi, dobimo ploskev imenovano katenoida.
    Je ena od dveh rotacijskih ploskev - druga je ravnina. Parametriziramo jo na naslednji način:
    \begin{align*}
        \phi : & \R^2 \to \R^3 \\
        & (t, \varphi) \mapsto \left( t, a \cdot \cosh \left( \frac{t}{a} \right) \cdot \cos\varphi , a \cdot \cosh \left( \frac{t}{a} \right) \cdot \sin\varphi \right)
    \end{align*}

    \begin{figure}[h]
        \includegraphics[height = 6cm]{katenoida.eps}
        \caption{Katenoida}
    \end{figure}

    Preverimo zdaj, da je katenoida res minimalna ploskev. To bomo dosegli z izračunom koeficientov
    prve in druge fundamentalne forme.

    Izračunajmo majprej vektorja hitrosti:
    \begin{align*}
        \phi_t &= \left( 1, \sinh\left( \frac{t}{a} \right) \cos\left(\varphi\right) , \sinh\left( \frac{t}{a} \right) \sin\left(\varphi\right) \right) \\
        \phi_\varphi &= \left( 0, - a \cosh\left( \frac{t}{a} \right) \sin\left(\varphi\right) , a \cosh\left( \frac{t}{a} \right) \cos\left(\varphi\right) \right)
    \end{align*}

    Sedaj lahko izračunamo ploskovno normalo:
    \begin{align*}
        \phi_t \times \phi_\varphi &= \left( a \sinh\left( \frac{t}{a} \right) \cosh\left( \frac{t}{a} \right) , - a \cos\left(\varphi\right) \cosh\left( \frac{t}{a} \right) , - a \sin\left(\varphi\right) \cosh\left( \frac{t}{a} \right) \right) \\
        \left\lvert \phi_t \times \phi_\varphi \right\rvert &= a \cosh^2 \left( \frac{t}{a} \right) \\
        \Rightarrow N &= \left( \tanh \left( \frac{t}{a} \right) , - \cos \left( \varphi \right) sech \left( \frac{t}{a} \right) , - \sin \left( \varphi \right) sech \left( \frac{t}{a} \right) \right)
    \end{align*}

    Izračunajmo še druge odvode parametrizacije $\phi$:
    \begin{align*}
        \phi_{tt} &= \left( 0, \frac{1}{a} \cosh\left( \frac{t}{a} \right) \cos\left(\varphi\right) , \frac{1}{a} \cosh\left( \frac{t}{a} \right) \sin\left(\varphi\right) \right) \\
        \phi_{t\varphi} &= \left( 0, - \sinh\left( \frac{t}{a} \right) \sin\left(\varphi\right) , \sinh\left( \frac{t}{a} \right) \cos\left(\varphi\right) \right) \\
        \phi_{\varphi\varphi} &= \left( 0, - a \cosh\left( \frac{t}{a} \right) \cos\left(\varphi\right) , - a \cosh\left( \frac{t}{a} \right) \sin\left(\varphi\right) \right) \\
    \end{align*}

    Sedaj imamo vse, da izračunamo koeficiente v prvi fundamentalni formi:
    \begin{align*}
        E &= I(\phi_t) = \cosh^2 \left( \frac{t}{a} \right) \\
        F &= I(\phi_t, \phi_{\varphi}) = 0 \\
        G &= I(\phi_{\varphi}) = a^2 \cosh^2 \left( \frac{t}{a} \right)
    \end{align*}

    in tudi v drugi fundamentalni formi:
    \begin{align*}
        l &= \phi_{tt} \cdot N = - \frac{1}{a} \\
        m &= \phi_{t\varphi} \cdot N = 0 \\
        n &= \phi_{\varphi\varphi} \cdot N = a
    \end{align*}

    Iz izračunanega sedaj sledi, da je srednja ukrivljenost enaka $0$:
    $$ H = En + Gl - 2Fm = a \cdot \cosh^2 \left( \frac{t}{a} \right) - a \cdot \cosh^2 \left( \frac{t}{a} \right) = 0$$

    To pa pomeni, da je katenoida res minimalna ploskev.
\end{primer}

\begin{primer}[Helikoid]
    Helikoid je ploskev, ki nastane s sočasnim vrtenjem in dvigovanjem daljice okrog neke osi.
    Ploskve, ki nastanejo na tak način, se imenujejo \emph{premostne ploskve}.
    Parametriziramo ga lahko kot:
    $$ \phi \left( u, v \right) = \left( v \cos \left( u\right), v \sin \left( u \right), u\right) $$

    \begin{figure}[h]
        \includegraphics[height = 6cm]{helikoid.eps}
        \caption{Helioid}
    \end{figure}

    Vektorja hitrosti sta:
    \begin{align*}
        \phi_u &= \left( - v \sin \left( u \right), v \cos \left( u \right), 1 \right) \\
        \phi_v &= \left( \cos \left( u \right), \sin \left( u \right), 0 \right)
    \end{align*}

    Izračunajmo normalo:
    \begin{align*}
        \phi_u \times \phi_v &= \left( - \sin(u), \cos(u), - v \right) \\
        \left\lvert \phi_u \times \phi_v \right\rvert &= v^2 \\
        \Rightarrow N &= \left( - \frac{1}{v^2} \sin(u), \frac{1}{v^2} \cos(u), - \frac{1}{v} \right)
    \end{align*}

    Drugi odvodi parametrizacije $\phi$:
    \begin{align*}
        \phi_{uu} &= \left( - v \cos(u), - v \sin(u), 0 \right) \\
        \phi_{uv} &= \left( - \sin(u), \cos(u), 0 \right) \\
        \phi_{vv} &= \left( 0, 0, 0 \right)
    \end{align*}

    Izračunajmo zdaj koeficiente v prvi in drugi fundamentalni formi:
    \begin{equation*}
        \begin{aligned}[c]
            E &= v^2 + 1 \\
            F &= 0 \\
            G &= 1
        \end{aligned}
        \qquad
        \begin{aligned}[c]
            l &= 0 \\
            m &= \frac{1}{v^2} \\
            n &= 0
        \end{aligned}
    \end{equation*}

    Vnesimo dobljene podatke v kriterij za minimalne ploskve:
    $$ H = En + Gl - 2Fm = 0 $$
    Ker je srednja ukrivljenost ničelna, helikoid res zadošča pogoju za minimalne ploskve.

\end{primer}

% Opr str. 82 
Katenoida in helikoid sta homeomorfna, saj lahko z deformacijo $\xi$ prehajamo med njima:
\begin{align*}
    \xi (t) = ( & \cos(t) \sin(u) \sinh(v) + \sin(t) \cos(u) \cosh(v), \\
        & - \cos(t) \cos(u) \sinh(v) + \sin(t) \sin(u) \sinh(v), \\
        & u \cos(t) + v \sin(t)  ) 
\end{align*}
Katenoido dobimo v primeru, ko je $t = \frac{\pi}{2}$, helikoid pa v primeru $t = 0$.

\begin{primer}[Enneperjeva ploskev]
    Enneperjeva ploskev je minimalna ploskev, ki seka samo sebe. Parametrično jo lahko podamo kot:
    \begin{align*}
        \phi : & \R^2 \to \R^3 \\
        & (r, \varphi) \mapsto \left( r \cos(\varphi) - \frac{1}{3} r^3 \cos(3 \varphi) , - \frac{1}{3} r \left( 3 \sin(\varphi) + r^2 \sin(3 \varphi), r^2 \cos(2 \varphi) \right) \right)
    \end{align*}

    Izračunamo koeficiente prve in druge fundamentalne forme za ta problem in dobimo:
    \begin{equation*}
        \begin{aligned}[c]
            E &= - 2 \cos(2 \varphi) \\
            F &= 4 r \cos(\varphi) \sin(\varphi) \\
            G &= 2 r^2 \cos(2 \varphi)
        \end{aligned}
        \qquad
        \begin{aligned}[c]
            l = (1 + r^2)^2 \\
            m = 0 \\
            n = r^2 (1 + r^2)^2
        \end{aligned}
    \end{equation*}

    Opazimo, da je sredna ukrivljenost ničelna $ H = 0 $, torej je Enneperjeva ploskev res minimalna.
\end{primer}

\begin{primer}[Prva Scherkova ploskev]
    Heinrich Scherk je odkril dve novi minimalni ploskvi leta 1834. Prva schreckova ploskev je dvojno 
    periodična, tj. če sta $u$ in $v$ periodi, potem je $f(z + u) = f(z + v) = f(z)$. Opišemo jo lahko 
    z implicitno enačbo:
    $$ e^z \cos(y) = \cos(x) $$
    Parametrično pa jo lahko opišemo kot:
    \begin{align*}
        \phi : & \R^2 \to \R^3 \\
        & (r, \varphi) \mapsto \left( ln \left( \frac{1 + r^2 + 2 r \cos(\varphi)}{1 + r^2 - 2 r \cos(\varphi)}\right), \frac{1 + r^2 - 2 r \sin(\varphi)}{1 + r^2 + 2 r \sin(\varphi)}, 2 \cdot tan^{-1} \left( \frac{2 r^2 \sin(2 \varphi)}{r^4 - 1 }\right) \right)
    \end{align*}

    \begin{figure}[h]
        \includegraphics[height = 6cm]{scherk1.eps}
        \caption{Del prve Scherkove ploskve}
    \end{figure}

    Kot prej lahko izračunamo, da je srednja ukrivljenost enaka nič.
\end{primer}

\begin{primer}[Druga Scherkova ploskev]
    Druga Scherkova minimalna ploskev je podana z enačbo
    $$ \sinh(x) * \sinh(y) - \sin(z) $$

    \begin{figure}[h]
        \includegraphics[height = 6cm]{scherk2.eps}
        \caption{Del druge Scherkove ploskve}
    \end{figure}
\end{primer}


\section{Enačba minimalne ploskve}

V tem poglavju bomo opisali ploskev, ki je graf funkcije v dveh spremenljivkah. Označimo jo z $z = f(x, y)$. Potem ploskev dobimo
s parametrizacijo $X (u, v) = (u, v, f (u, v))$.

Vektorja hitrosti sta potem 
$$ X_u = (1, 0, f_u) \qquad \qquad X_v = (0, 1, f_v)  $$ 

Drugi odvodi so torej 
$$
    X_{uu} = (0, 0, f_{uu}) \qquad \qquad
    X_{uv} = (0, 0, f_{uv}) \qquad \qquad
    X_{vv} = (0, 0, f_{vv})
$$

Iz tega lahko dobimo enotsko normalo $N = \frac{(- f_u, - f_v, 1)}{\sqrt{1 + f_u^2 + f_v^2}}$.

Želimo izvedeti pri kakšnih pogojih je definirana ploskev minimalna. Zato želimo izračunati koeficiente iz prve in druge fundamentalne forme.
Dobimo sledeče podatke:

$$ E = 1 + f_u^2 \qquad 
F = f_u \cdot f_v \qquad 
G = 1 + f_v^2 $$

$$ l = \frac{f_{uu}}{\sqrt{1 + f_u^2 + f_v^2}} \qquad 
m = \frac{f_{uv}}{\sqrt{1 + f_u^2 + f_v^2}} \qquad
n = \frac{f_{vv}}{\sqrt{1 + f_u^2 + f_v^2}} $$

Izračunamo, da je srednja ukrivljenost 
\begin{align*}
    H &= E n + G l - 2 F m \\
    &= \frac{ (1 + f_v^2) f_{uu} - 2 f_u f_v f_{uv} + (1 + f_u^2) f_{vv} }{ 2 (1 + f_u^2 + f_v^2)^{\frac{3}{2}} }
\end{align*} 

Vemo, da je ploskev minimalna natanko takrat, ko je njena srednja ukrivljenjost ničelna. Iz tega sledi naslednja 
predpostavka:

\begin{trditev}
    Ploskev $M$ podana kot graf funkcije $z = f(x, y)$ je minimalna natanko tedaj, ko je
    $$ (1 + f_v^2) f_{uu} - 2 f_u f_v f_{uv} + (1 + f_u^2) f_{vv} = 0 $$
\end{trditev}

Enačbo iz trditve imenujemo enačba minimalne ploskve. V splošnem ni rešljiva, z dodatnimi robnimi pogoji, pa jo lahko rešimo in
determiniramo različne tipe minimalnih ploskev.

En izmed primerov, ki ga lahko rešimo z uporabo enačbe minimalne ploskve je Scherkova prva minimalna ploskev. 

\begin{primer}
    % ----------------------------------------------------------------------------------------------------
\end{primer}



\section{Izotermne koordinate}
Minimalno ploskev lahko definiramo tudi s pomočjo pojma harmoničnosti funkcije in parametrizacije, ki ji pravimo izotermna.

\begin{definicija}
    Parametrizacije ploskve $phi$ je \emph{izotermna}, če velja $$ E = \phi_u \cdot \phi_u = \phi_v \cdot \phi_v = G \qquad \text{ in } \qquad F = \phi_u \cdot \phi_v = 0 $$ 
\end{definicija}

Ko ima neka ploskev izotermne koordinate, se njena srednja ukrivljenost posploši na $H = \frac{n + l}{2E}$,
kjer so $n, l \text{ in } E$ ravno koeficienti iz prve in druge fundamentalne forme. 
To pa pomeni, da je ploskev z izotermno parametrizacijo minimalna natanko tedaj, ko je $n + l = 0$. 

Izkaže se, da izotermne koordinate obstajajo za vsako ploskev. Ker nas zanimajo predvsem minimalne ploskve,
pride v poštev naslednja lema.

\begin{lema}
    Za vsako minimalno ploskev $M$ obstajajo izotermne koordinate.
\end{lema}

\begin{dokaz}
    Naj bo $m \in M$ točka in določimo koordinatni sistem tako, da bo $m$ ravno njegovo izhodišče. 
    Koordinatni sistem naj bo obrnjen tako, da je tangentna ravnina $T_m M$ natanko ravnina $z = 0$.
    Ploskev $M$ lahko potem podamo kot graf funkcije $z = f(x, y)$.

    Izračunajmo naslednja izraza:
    \begin{align*}
        \left( \frac{1 + f_x^2}{w} \right)_y - \left( \frac{f_x f_y}{w} \right)_x &= - \frac{f_y}{w} \left[ f_{xx} (1 + f_y^2) - 2 f_x f_y f_{xy} + f_{yy} (1 + f_x^2) \right] \\
        \left( \frac{1 + f_y^2}{w} \right)_x - \left( \frac{f_x f_y}{w} \right)_y &= - \frac{f_x}{w} \left[ f_{xx} (1 + f_y^2) - 2 f_x f_y f_{xy} + f_{yy} (1 + f_x^2) \right]
    \end{align*}

    kjer je $w = \sqrt{1 + f_x^2 + f_y^2}$

    Na desni strani prepoznamo enačbo za minimalne ploskve. Ker je $M$ minimalna, sta torej oba izraza na desni strani enaka nič.
    Označimo še $p = f_x$ in $q = f_y$. Sedaj sledita naslednji enakosti:
    $$ \left( \frac{1 + p^2}{w} \right)_y - \left( \frac{pq}{w} \right)_x = 0 \qquad \left( \frac{1 + q^2}{w} \right)_x - \left( \frac{pq}{w} \right)_y = 0 $$.

    Definirajmo naslednja vektorska prostora:
    $$ V = \left( \frac{1 + p^2}{w} , \frac{pq}{w} \right)  \qquad\qquad W = \left( \frac{1 + q^2}{w} , \frac{pq}{w} \right) $$

    Na definiranih vektorskih prostorih uporabimo Greenovo formulo in dobimo:
    \begin{align*}
        \int_{C} V (dx, dy) &= \int\int_R \left( \frac{pq}{w} \right)_x - \left( \frac{1 + p^2}{w} \right)_y dx dy = 0 \\
        \int_{C} W (dx, dy) &= \int\int_R \left( \frac{1 + p^2}{w} \right)_x - \left( \frac{pq}{w} \right)_y  dx dy = 0 
    \end{align*}
    kjer je $R$ območje z robom $C$, ki je pozitivno orientirana sklenjena krivulja.

    Iz vektorske analize vemo, da je integral potencialnega polja po sklenjeni usmerjeni krivulji ničeln. To pomeni, da sta vektorski
    polji $V$ in $W$ potencialni, torej po definiciji potencialneg polja obstajata taki skalarni funkciji $\mu$ in $\rho$, da je $V = grad(\mu)$ in $W = grad(\rho)$.
    Zanju veljajo naslednje zveze:
    \begin{equation*}
        \begin{aligned}[c]
            \mu_x &= \frac{1 + p^2}{w} \\
            \mu_y &= \frac{pq}{w}
        \end{aligned}
        \qquad
        \begin{aligned}[c]
            \rho_x &= \frac{pq}{w} \\
            \rho_y &= \frac{1 + q^2}{w}
        \end{aligned}
    \end{equation*} 

    Definirajmo preslikavo 
    \begin{align*}
        T : & R \to \R^2 \\
        & (x, y) \mapsto \left( x + \mu (x, y) , y + \rho (x, y) y\right) 
    \end{align*}

    Izračunajmo Jacobijevo matriko preslikave $T$:

    $$ J(T) = 
    \left[
        \begin{matrix}
            1 + \mu_x & \mu_y \\
            \rho_x & 1 + \rho_y
        \end{matrix}
    \right]
     = 
    \left[
        \begin{matrix}
            1 + \frac{1 + p^2}{w} & \frac{pq}{w} \\
            \frac{pq}{w} & 1 + \frac{1 + q^2}{w}
        \end{matrix}
    \right]
    $$

    Determinanta Jacobijeve matrike $J(T)$ ni ničelna, zato po izreku inverzne funkcije obstaja 
    inverzna preslikava $T^{-1} (u, v) = (x, y)$ v okolici točke $m = (0, 0)$

    $$
        J(T^{-1}) = J(T)^{-1}
        = 
            \frac{1}{det(J(T))} 
            \left[
            \begin{matrix}
                1 + \frac{1 + p^2}{w} & \frac{pq}{w} \\
                \frac{pq}{w} & 1 + \frac{1 + q^2}{w}
            \end{matrix} 
        \right]
        = 
            \frac{1}{(1 + w)^2}
            \left[
            \begin{matrix}
                1 + \frac{1 + p^2}{w} & - \frac{pq}{w} \\
                - \frac{pq}{w} & 1 + \frac{1 + q^2}{w}
            \end{matrix}
        \right]
        = 
        \left[
            \begin{matrix}
                x_u & x_v \\
                y_u & y_v
            \end{matrix}
        \right]
    $$

    Iz vsega dobljenega bomo sedaj pokazali, da je parametrizacija 
    $$ \xi (u, v) = \left( x(u,v), y(u, v), f(x(u, v), y(u, v)) \right) $$
    izotermna parametrizacije minimalne ploskve $M$. 

    Ker smo koordinatni sistem postavili tako, da je $M$ ravno graf preslikave $f$, 
    $\xi$ res parametrizira našo ploskev. Pokažimo še, da zadošča pogojem izotermnosti.

    $$ \xi_u = \left( \frac{w^2 + 1 + q^2}{(1 + w)^2} , - \frac{pq}{(1 + w)^2} , 
    p \left( \frac{w^2 + 1 + q^2}{(1 + w)^2} \right) + q \left( - \frac{pq}{(1 + w)^2} \right) \right) $$

    $$ \xi_v = \left( - \frac{pq}{(1 + w)^2} , \frac{w^2 + 1 + q^2}{(1 + w)^2} , 
    p \left( - \frac{pq}{(1 + w)^2} \right) + g \left( \frac{w^2 + 1 + q^2}{(1 + w)^2} \right)\right) $$

    Sledi, da je $E = \xi_u \cdot \xi_u = \xi_v \cdot \xi_v = G$ in $F = \xi_u \cdot \xi_v = 0$.
\end{dokaz}

Ker sedaj vemo, da izotermne koordinate za vsako minimalno ploskev res obstajajo, lahko navedemo 
še definicijo minimalne ploskve s harmoničnostjo njene parametrizacije.

\begin{trditev}
    Naj bo $M$ regularna ploskev z izotermno parametrizacijo $\phi$. 
    Velja, da je $M$ minimalna natanko tedaj, ko je $\phi$ harmonična.
\end{trditev}

\begin{dokaz}
    Vemo, da je $\phi$ izotermna, torej velja enakost $H = \frac{n + l}{2E}$. 
    Iz tega sledi 
    $$ 2 E H = n + l = \phi_{uu} \cdot N + \phi_{vv} \cdot N = N \cdot \left( \phi_{uu} + \phi_{vv} \right) = N \cdot \Delta \phi $$

    Če je $M$ minimalna, potem je $H = 0$, zato je $N \cdot \Delta \phi = 0$.
    $N$ je enotska normala, torej dolžine $\left\lvert N \right\rvert  = 1$, zato je $\Delta \phi = 0$,
    kar pa je ravno definicija harmonične funkcije.

    Naj bo sedaj parametrizacija $\phi$ harmonična. Potem je $2 E H = 0$, saj $\left\lvert N \right\rvert  = 1$.
    Ker je ploskev regularna, $E \neq 0$, zato pa je $H = 0$.
\end{dokaz}


\section{Weierstrass-Enneperjeva reprezentacija}
Do zdaj smo ugotovili, da se da vsako minimalno ploskev parametrizirati. V tem poglavju bomo poiskali to parametrizacijo.
Naj bo $\phi (u, v)$ izotermna parametrizacija minimalne ploskve $M$ in označimo kompleksno koordinato z $z$.
Vemo, da lahko $u$ in $v$ izrazimo z $z$ in njeno konjungiranko $\bar{z}$ kot:
$$ u = \frac{z + \bar{z}}{2} \qquad\qquad v = \frac{z - \bar{z}}{2i} $$

Zato lahko našo parametrizacijo zapišemo v odvisnosti od $z$ in $\bar{z}$ kot 
$\phi(z, \bar{z}) = \left( \varphi^1(z, \bar{z}) , \varphi^2(z, \bar{z}) , \varphi^3(z, \bar{z}) \right)$,
kjer so $\varphi^{i}$ kompleksne funkcije, ki zasedajo realne vrednosti.

Odvod naše parametrizacije je $\phi_z = \left( \varphi^{1}_{z} , \varphi^{2}_{z} , \varphi^{3}_{z} \right)$

Označimo sedaj:
\begin{align*}
    \left( \phi_z \right)^2 &= (\varphi^{1}_{z})^2 + (\varphi^{2}_{z})^2 + (\varphi^{3}_{z})^2  \\
    \left\lvert \phi_z \right\rvert^2 &= \left\lvert \varphi^{1}_{z} \right\rvert^2 + \left\lvert \varphi^{2}_{z} \right\rvert^2 + \left\lvert \varphi^{3}_{z} \right\rvert^2 
\end{align*}

V nadaljevanju bomo pokazali povezavo med $\left( \phi_z \right)^2$ oziroma $\left\lvert \phi_z \right\rvert^2$ in 
izotemnostjo preslikave $\phi$.
Zavoljo tega bi najprej radi izračunali $\left( \phi_z \right)^2$ in $\left\lvert \phi_z \right\rvert^2$,
torej tudi $\left( \varphi^{i}_z \right)^2$ oziroma $\left\lvert \varphi^{i}_z \right\rvert^2$.

Po verižnem pravilu velja $ \frac{\delta \varphi^{i}}{\delta z} = \frac{\delta \varphi^{i}}{\delta u} \frac{\delta u}{\delta z} + 
\frac{\delta \varphi^{i}}{\delta v} \frac{\delta v}{\delta z} = \frac{1}{2} \left( \frac{\delta \varphi^{i}}{\delta u} - i \frac{\delta \varphi^{i}}{\delta v}\right)$

Krajše zapisano je to $\varphi^{i}_z = \frac{1}{2} \left( \varphi^{i}_u - i \varphi^{i}_v \right)$

\begin{align*}
    \left( \varphi^{i}_z \right)^2 &= \left( \frac{1}{2} (\varphi^{i}_u - i \varphi^{i}_v)\right)^2 \\
    &= \frac{1}{4} \left( (\varphi^{i}_u)^2 - (\varphi^{i}_v)^2 - 2i \varphi^{i}_u \varphi^{i}_v \right) 
\end{align*}

Torej je

\begin{align*}
    \left( \phi_z \right)^2 &= (\varphi^{1}_{z})^2 + (\varphi^{2}_{z})^2 + (\varphi^{3}_{z})^2  \\
    &= \frac{1}{4} \left( \sum_{j = 1}^{3} (\varphi^{j}_u)^2 - \sum_{j = 1}^{3} (\varphi^{j}_v)^2 - 2i \sum_{j = 1}^{3} \varphi^{j}_u \varphi^{j}_v \right) \\
    &= \frac{1}{4} \left( \left\lvert \phi_u \right\rvert^2 - \left\lvert \phi_v \right\rvert^2 - 2i (\phi_u \cdot \phi_v) \right) \\
    &= \frac{1}{4} \left( E - G - 2i F \right)
\end{align*}

Sedaj lahko pokažemo naslednjo lemo.

\begin{lema}
    $\phi$ ima izotermne koordinate natanko tedaj, ko je $\left( \phi_z \right)^2 = 0$.
\end{lema}

\begin{dokaz}
    Naj bo $\phi$ izotermna parametrizacija. Potem je $\left( \phi_z \right)^2 = \frac{1}{4} \left( E - G - 2i F \right) 
    = \frac{1}{4} \left( E - E - 2i \cdot 0 \right) = 0$

    Obratno, naj bo $\left( \phi_z \right)^2 = 0$. Potem je $\frac{1}{4} \left( E - G - 2i F \right) = \frac{E}{4} - \frac{G}{4} - i \frac{F}{2} = 0 + i \cdot 0$.
    Dobimo sistem dveh enačb $\frac{E}{4} - \frac{G}{4} = 0$ in $\frac{F}{2} = 0$ iz česar seveda sledi $E = G$ in $F = 0$. Tako je naša lema dokazana.
\end{dokaz}

Izračunajmo še kvadrat absolutne vrednosti odvoda parametrizacije in najdimo njegovo povezavo z izotermnostjo.


$$ \left\lvert \varphi^{i}_z \right\rvert^2 = \frac{1}{4} \left( \left( \varphi^{i}_u \right)^2 + \left( \varphi^{i}_v \right)^2 \right) $$

Zato velja

\begin{align*}
    \left\lvert \phi_z \right\rvert^2 &= \left\lvert \varphi^{1}_z \right\rvert^2 + \left\lvert \varphi^{2}_z \right\rvert^2 + \left\lvert \varphi^{3}_z \right\rvert^2 \\
    &= \frac{1}{4} \left( \sum_{j = 1}^{3} \left\lvert \varphi^{j}_u \right\rvert^2 + \sum_{j = 1}^{3} \left\lvert \varphi^{j}_v \right\rvert^2 \right) \\
    &= \frac{1}{4} \left( \left\lvert \phi_u \right\rvert^2 + \left\lvert \phi_v \right\rvert^2 \right) \\
    &= \frac{1}{4} \left( E + G \right)
\end{align*}

Iz izračunanega sedaj sledi lema:

\begin{lema}
    Če je $\phi$ regularna parametrizacija minimalne ploskve, potem je $\left\lvert \phi_z \right\rvert^2 = \frac{E}{2} \neq 0$
\end{lema}

\begin{dokaz}
    Naj bo $\phi$ izotermna. Potem je $E = G$, in zato $\left\lvert \phi_z \right\rvert^2 = \frac{E + E}{4} = \frac{E}{2}$. 
    Ker je $\phi$ tudi regularna, velja $E \neq 0$
\end{dokaz}


Potrebovali bomo še dve lemi, ki veljata v splošnem za kompleksne funkcije. 
Označimo to funkcijo z $f (z) = r (u, v) + i s (u, v)$, kjer sta $r$ in $s$ realni funkciji realnih spremenljivk.

\begin{lema}
    $f$ je holomorfna natanko tedaj, ko velja enakost $ \frac{\delta f}{\delta \bar{z}} = 0$.
\end{lema}

\begin{dokaz}
    Naj bo najprej $f$ holomorfna funkcija. Potem izračunamo:
    \begin{align*}
        \frac{\delta f}{\delta \bar{z}} &= \frac{1}{2} \left( \frac{\delta f}{\delta u} + i \frac{\delta f}{\delta v} \right) \\
        &= \frac{1}{2} \left( \frac{\delta r}{\delta u} + i \frac{\delta s}{\delta u} + i \left( \frac{\delta r}{\delta v} + \frac{\delta s}{\delta v} \right) \right) \\
        &= \frac{1}{2} \left( \frac{\delta r}{\delta u} + i \frac{\delta s}{\delta u} - i \frac{\delta s}{\delta u} - \frac{\delta r}{\delta u} \right) \\
        &= 0
    \end{align*}
    Tukaj smo v tretji enakosti uporabili Cauchy - Riemannova pogoja.
    
\end{dokaz}

\begin{lema}
    Velja enakost $\Delta f = f_{uu} + f_{vv} = 4 \left( \frac{1}{\delta z} \left( \frac{\delta f}{\delta \bar{z}} \right)\right)$
\end{lema}

\begin{dokaz}
    Računamo:
    \begin{align*}
        \frac{\delta}{\delta z} \left( \frac{\delta f}{\delta \bar{z}} \right) &=
        \frac{\delta}{\delta z} \left( \frac{1}{2} \left( f_u + i f_v \right) \right) \\
        &= \frac{1}{4} \left( f_{uu} + f_{vv} \right) \\
        &= \frac{1}{4} \Delta f
    \end{align*}
    Enakost je dokazana.
\end{dokaz}


Pripravili smo vse, da lahko navedemo še zadnjo definicijo minimalne ploskve.

\begin{trditev}
    Naj bo $M$ ploskev z izotermno parametrizacijo $\phi$. Velja, da je $M$ minimalna, če in samo če je $\phi'$ holomorfna.
\end{trditev}

\begin{dokaz}
    V trditvi (4.3) smo že povedali, da je izotermna parametrizacija minimalne ploskve harmonična.
    Dodatno po lemi () sledi zveza:
    $$ 4 \Delta \phi = 0 = \frac{\delta}{\delta \bar{z}} \left( \frac{\delta \phi}{\delta z} \right) = \frac{\delta \phi'}{\delta \bar{z}} $$

    Po lemi () pa še velja, da je $\phi'$ holomorfna natanko tedaj kadar je $\frac{\delta \phi'}{\delta \bar{z}} = 0$.
    Ker povsod v zgornjih izjavah veljajo ekvavilence, smo dokazali v obe smeri.
\end{dokaz}

Če je torej $\phi' = \left( \varphi^1_z, \varphi^2_z, \varphi^3_z \right)$ holomorfna funkcija za katero velja
$\left( \phi \right)^2 = 0$, je po lemi (5.1) izotermna parametrizacija in zato po ravnokar dokazani trditvi 
parametrizacija minimalne ploskve.
Minimalno ploskev lahko torej zapišemo kot trojico holomorfnih funkcij, parametrizacijo $\phi (z, \bar{z}) = (\varphi^1, \varphi^2, \varphi^3)$
pa lahko zapišemo eksplicitno.

Poglejmo kako.

Poznamo že zvezi $\varphi_z^{i} = \frac{1}{2} \left( \varphi_u^{i} - i \varphi_v^{i} \right)$ in $dz = du + i dv$ ter $d\bar{z} = du - i dv$.
\begin{align*}
    d \varphi^{i} &= d \varphi^{i}_z dz + d \varphi^{i}_\bar{z} d\bar{z}   \\
    &= \frac{1}{2} \left( \varphi^{i}_u - i \varphi^{i}_v \right) \left( du + i dv \right) + 
    \frac{1}{2} \left( \varphi^{i}_u + i \varphi^{i}_v \right) \left( du - i dv \right)   \\
    &= \varphi^{i}_u + i \varphi^{i}_v   \\
    &= 2 Re \left( \varphi^{i}_z dz \right)
\end{align*}

Integriramo in dobimo:
$$ \varphi^{i} (z, \bar{z}) = c_i + 2 Re \left( \int \varphi_z^{i} dz \right) $$

Sedaj bi radi še poiskali trojice holomorfnih funkcij $\phi' = \left( \varphi^1_z, \varphi^2_z, \varphi^3_z \right)$,
ki res zadoščajo pogoju $\left( \phi \right)^2 = 0$.
Za to sta poskrbela Weierstrass in Enneper in podala kar dve reprezentaciji minimalne ploskve.

\begin{izrek}[Weierstrass - Enneperjeva reprezentacija $I$]
    Če je $f$ holomorfna na domeni $D$ in $g$ meromorfna na $D$ tako, da je $fg^2$ holomorfna na $D$,
    potem je minimalna ploskev definirana s parametrizacijo $\phi (z, \bar{z}) = (\varphi^1 (z, \bar{z}), \varphi^2 (z, \bar{z}), \varphi^3 (z, \bar{z}))$,
    v kateri so:
    \begin{align*}
        \varphi^1 &= Re \left( \int f (1 - g^2) dz \right) \\
        \varphi^2 &= Re \left( \int i f (1 + g^2) dz \right) \\
        \varphi^3 &= 2 \cdot Re \left( \int f g dz \right)
    \end{align*}
\end{izrek}

\begin{dokaz}
    Pokazati moramo, da je $\phi' = \left( \varphi^1_z, \varphi^2_z, \varphi^3_z \right)$ trojica holomorfnih
    funkcij, za katero velja $\left( \phi \right)^2 = 0$.

    $\varphi^1_z, \varphi^2_z \text{ in } \varphi^3_z$ so res holomorfne, saj jih dobimo iz holomorfnih funkicj z uporabo osnovnih operacij.

    Preverimo še pogoj izotermne parametrizacije.
    \begin{align*}
        (\phi')^2 &= \left( \frac{1}{2} f (1 - g^2) \right)^2  + \left( \frac{1}{2} i f (1 + g^2) \right) + \left( fg \right)^2 = \\
        &= \frac{1}{4} f^2 \left( - 4 g^2 \right) + f^2 g^2 = \\
        &= f^2 g^2 - f^2 g^2 = \\
        &= 0
    \end{align*}
\end{dokaz}


Pogoje v izreku bi radi zamenjali z enim samim, zato denimo, da je $f$ holomorfna in  $g$ holomorfna s holomorfnim inverzom $g^{-1}$.
Definirajmo novo spremenljivko $\tau = g$. Potem je $d \tau = g' dg$.
Definirajmo še funkcijo $F(\tau) = \frac{f}{g'}$. Velja $F(\tau) d\tau = f d\tau$.

V izreku lahko torej zamenjamo $g$ za $\tau$ in $f dz$ za $F(\tau) d\tau$ in dobimo:

\begin{izrek}[Weierstrass - Enneperjeva reprezentacija $II$]
    Za holomorfno funkcijo $F(\tau)$ je minimalna ploskev definirana s parametrizacijo 
    $\phi = (\varphi^1, \varphi^2, \varphi^3)$, kjer je 
    \begin{align*}
        \varphi^1 &= Re \left( \int (1 - \tau^2) F(\tau) d\tau \right) \\
        \varphi^2 &= Re \left( \int i (1 - \tau^2) F(\tau) \right) \\
        \varphi^3 &= 2 \cdot Re \left( \tau F(\tau) \right)
    \end{align*}
\end{izrek}



\section{Princip identičnosti}

To poglavje je namenjeno izpeljavi izreka, ki je nujen za dokaz Björlingovega problema in njegovih posledic.
Pove nam kdaj lahko dve funkciji identificiramo kot enaki na nekem območju.

\begin{izrek}
    Naj bosta $f$ in $g$ holomorfni funkciji na povezanem odprtem območju $D \subseteq \mathbb{C} (ali \mathbb{R})$.
    Če ima $S = { z \in D | f(z) = g(z) }$ stekališče, potem je $f = g$ na $D$.
\end{izrek}

Da lahko izrek dokažemo, potrebujemo dodatno lemo.

\begin{lema}
    Naj bo $f$ holomorfna na $D \in \mathbb{C}$ in $L$ množica stekališč množice ${ z \in D | f(z) = 0 } \subseteq D$.
    Potem je $L$ zaprta in odprta v $D$.
\end{lema}

\begin{dokaz}
    Po definiciji stekališča je $L$ zaprta množica.

    Naj bo $z_o \in L$ in izberimo odprto kroglo $K (z_o, r) \subseteq D$.
    Zapišimo $ f(z) = \sum_{n = 0}^{\infty} a_n (z - z_o)^n $, kjer je $z \in K (z_o, r)$.
    Potem je $f(z_o) = 0$. Zato ima bodisi $f$ ničlo $z_o$ reda $m$, za $m \in \mathbb{N}$ bodisi je $a_n = 0$ za vsak $n \in \mathbb{N}$.

    V primeru, da ima $f$ ničlo reda $m$, obstaja taka analitična funkcija $g$, da je $f(z) = (z - z_o)^m  g(z)$ za $z \in D$,
    kjer $g(z_o) \neq 0$. Ker je $g$ zvezna, je $g(z) \neg 0$ za $z$ blizu $z_o$. Posledično je $z_o$ izolirna točka množice
    ${ z \in D | f(z) = 0 }$. Torej velja $z_o \notin L$. Ker ta točka ne velja, mora veljati $a_n = 0$ za vsak $n \in \mathbb{N}$.

    V kolikor za vsak $n \in \mathbb{N}$ velja $a_n = 0$, je $f(z) = 0$ za vse točke v krogli $z in K(z_o, r)$. Zato je krogla $K(z_o, r)$ 
    vsebovana v množici $L$. Posledično je $L$ odprta množica.
\end{dokaz}


S pomočjo leme lahko sedaj dokažemo še princip identitete.

\begin{dokaz}
    Po predpostavki vemo, da ima množica $S = { z \in D | f(z) - g(z) = 0 }$ stekališe.
    Zato je po predpostavki zaprta in odprta. Vemo tudi, da je množica $D$ povezana. Edina podmnožica 
    povezane množice, ki je hkrati odprta in zaprta, pa je množica sama. 

    Torej je $D = { z \in D | f(z) = g(z) }$ in zato identificiramo $f = g$ na množici $D$.
\end{dokaz}



\section{Björlingov problem}

Da lahko problem navedemo, moramo poznati definicijo realno analitične funkcije.

\begin{definicija}
    Funkcija $f(x)$ realne spremenljivke $x$ je \emph{realno analitična}, če je $f(z)$ 
    holomorfna za kompleksno spremenljivko $z$.
\end{definicija}

Ekvavilentno bi lahko realno analitično funkcijo definirali kot funkcijo, katere Taylorjeva
vrsta konvergira k njej sami. $f(z)$ se imenuje \emph{holomorfna razširitev} funkcije $f(x)$.


Recimo sedaj, da je $\alpha (t) : I \mapsto \R^3$ realno analitična krivulja in $\eta : I \mapsto \R^3$ 
realno analitično vektorsko polje, da velja $ \left\lvert \eta \right\rvert = 1$ in $\eta (t) \cdot \alpha' (t) = 0$ 
za vse točke $t$ iz intervala $I$. Drugi pogoj pove, da je $\eta$ pravokoten na vse tangentne vektorje krivulje $\alpha$.

Björlingov problem zahteva, da najdemo takšno parametrizacijo $\phi (u, v)$ minimalne ploskve $M$, za katero velja:
\begin{enumerate}
    \item Ploskev $M$ naj vsebuje krivuljo $\alpha$ pri $v = 0$. Tj. $\forall u \in I. \alpha (u) = \phi (u, 0)$
    \item Normale na ploskev $M$ se naj vzdolž celotne krivulje $\alpha$ ujemajo z vektorji vektorskega polja $\eta$ 
    Tj. $\forall u \in I. \eta (u) = \mathcal{N} (u, 0)$
\end{enumerate}

Ker sta $\alpha(t)$ in $\eta(t)$ realno analitični funkciji, sta njuni holomorfni funkciji $\alpha(z)$ 
in $\eta(z)$ kompleksni holomorfni funkcji na $D \mapsto \C^3$, kjer je $I \subseteq D \subseteq \C$.

Izkaže se, da je rešitev Björlingovega problema ena sama in jo lahko zapišemo eksplicitno:
$$ \phi (u, v) = Re \left( \alpha (z) - i \int_{z_0}^{z} \eta (w) \times \alpha' (w) dw \right) $$

Zgornjo izjavo bomo dokazali v dveh korakih. Najprej bomo privzeli, da je $\phi$ rešitev in pokazali,
da je takšna kot v izreku. Iz tega bo sledilo, da je vsaka rešitev Björlingovega problema takšne oblike 
kot smo jo navedli in je torej enolična. Nato bomo še pokazali, da rešitvena parametrizacija zadošča 
obema pogojema Björlingovega problema, kar pa pomeni, da res obstaja.


\begin{dokaz}
    Recimo, da je $\phi$ rešitev Björlingovega problema.
    $\phi$ torej parametrizira minimalno ploskev $M$.
    Ker za vsako minimalno ploskev obstaja izotermna parametrizacija, predpostavimo, da je $\phi$ izotermna.
    Po trditvi (4.3) je $\phi$ harmonična: $\Delta \phi = 0$.

    Naj bo $\varphi^{j}$ harmonična konjungiranka $\phi^{j}$ tako, da je $\phi^{i} + i \varphi^{i}$ holomorfna.
    Definirajmo holomorfno funkcijo
    \begin{align*}
        & \beta (z) : D \mapsto \C^3 \\
        & \beta (z) = \phi (z) + i \varphi (z) = \left[
        \begin{matrix}
            \phi^{1} \\
            \phi^{2} \\
            \phi^{3} \\
        \end{matrix} \right]
        + i \left[
        \begin{matrix}
            \varphi^{1} \\
            \varphi^{2} \\
            \varphi^{3} \\
        \end{matrix} \right].
    \end{align*}

    $\phi$ in $\varphi$ sta funkciji v realnih spremenljivkah $u$ in $v$, zato lahko tudi $\beta$ odvajamo po $u$ in $v$.
    Odvod $\beta$ po $u$ je $\frac{\delta}{\delta u} \beta(z) = \frac{\delta}{\delta u} (\phi + i \varphi) = \phi_u + i \varphi_u$.
    Ker je $\beta$ holomorfna, ustreza Cauchy-Riemannovim pogojem, zato lahko odvod zapišemo kot $\frac{\delta}{\delta u} \beta(z) = \phi_u - i \phi_v$.

    Predpostavili smo, da je $phi$ izotermna. Po definiciji izotermnosti velja, da sta si vektorja hitrosti $\phi_u$ in $\phi_v$
    pravokotna. Normala $N$ na ploskev pa je pravokotna na oba vektorja hitrosti, zato je $\phi_v = N \times \phi_u$.

    Sledi $\beta'(z) = \phi_u(z) - i \cdot \left( N(z) \times \phi_u(z) \right)$.
    Če se sedaj omejimo le na realne vrednosti $z$, dobimo 
    $$\beta'(u) = \alpha'(u) - i \cdot \left( \mathcal{N} (u) \times \alpha'(u) \right)$$
    za $u \in I$, kjer je $I$ kot v pogojih Björlingovega problema. Enakost velja, ker je $\phi(u, 0) = \alpha(u)$ in $N(u, 0) = \mathcal{N} (u)$.

    Izraz lahko integriramo po realni spremenljivki $u$ in dobimo:
    $$\beta(u) = \alpha(u) - i \cdot \int_{u_0}^{u} \mathcal{N} (t) \times \alpha'(t) dt$$.

    Recimo sedaj, da je $\gamma(z) = \alpha(z) - i \cdot \int_{u_0}^{z} \mathcal{N} (w) \times \alpha'(w) dw$ holomorfna krivlulja.
    Opazimo, da je ta vsak $u \in I, \beta(u) = \gamma(u + i \cdot 0)$. Zato po izreku identitete velja $\beta(z) = \gamma(z)$ za vsak $z \in D$.
    
    Realni del funkcije $\beta$ je ravno parametrizacija $\phi = Re(\phi + i \cdot \varphi)$. Torej je 
    $$ \phi(u, v) = Re(\beta(z)) = Re(\alpha(z) - i \cdot \int_{u_0}^{z} \mathcal{N} (w) \times \alpha'(w) dw)$$,
    kar je pa ravno rešitev problema.

    Dokazali smo, da če $\phi$ parametrizira ploskev in zadošča pogojem problema, potem je ravno takšne oblike kot je 
    navedeno v rešitvi. To pomeni, da je rešitev enolična.

    Za dokaz obstoja, naj bo $\beta(z)$ holomorna funkcija definirana kot 
    \begin{align*}
        \beta (z) &= \phi(z) + i \cdot \varphi(z) \\
        &= \alpha(z) - i \cdot \int_{u_0}^{z} \mathcal{N} (w) \times \alpha'(w) dw
    \end{align*}

    Naj bo $u \in I$. Vemo, da sta $\alpha'(u)$ in $\mathcal{N} (u)$ realni. Omejimo sedaj funkcijo $\beta$ na $u$. 
    \begin{align*}
        &\beta(u) = \alpha(u) - i \cdot \int_{u_0}^{u} \mathcal{N} (t) \times \alpha'(t) dt \\
        \Rightarrow \qquad &\beta'(u) = \alpha'(u) - i \cdot \left( \mathcal{N} (u) \times \alpha'(u) \right)
    \end{align*}

    Vemo, da je $\mathcal{N} \times \alpha' \bot \alpha'$, zato je $\left( \mathcal{N} \times \alpha' \right) \cdot \alpha'$.

    Prav tako vemo, da je $\mathcal{N} \bot \alpha'$, zato velja $|\mathcal{N} \times \alpha'| = |\mathcal{N}| \cdot |\alpha'| = |\alpha'|$.

    Izračunajmo:
    \begin{align*}
        \beta'(u)^2 &= \left( \alpha'(u) - i \cdot \left( \mathcal{N} (u) \times \alpha'(u) \right) \right)^2 
        &= \alpha'(u) \cdot \alpha'(u) - 2i \alpha'(u) \cdot (\mathcal{N} \times \alpha'(u)) - (\mathcal{N} \times \alpha'(u)) \cdot (\mathcal{N} \times \alpha'(u))
        &= |\alpha'(u)|^2 - 0 - |\mathcal{N} \times \alpha'(u)|^2
        &= |\alpha'(u)|^2 - |\alpha'(u)|^2
        &= 0
    \end{align*}

    Po izreku identitete iz $\beta'(u)^2 = 0$ za $\forall u \in I$ sledi $\beta'(u)^2 = 0$ za $\forall u \in D$.
    To pa je ravno pogoj za obstoj izotermnih koordinat po lemi (7.1).
    Torej realni del funkcije $\beta$ parametrizira minimalno ploskev $M$ v izotermnih koordinatah.
    Tako smo dokazali obstoj izotermne parametrizacije ploskve, ki reši Björlingov problem.
\end{dokaz} 



Predpostavimo sedaj, da imamo naravno parametrizacijo krivulje $\alpha(s)$, ki ima enotski tangentni vektor $T = \alpha'(s)$.
Naj bo $N = \frac{T'}{| T' |} = \frac{\alpha''}{| \alpha |}$ njena enotska normala in $B = T \times N$ njena stranska normala.
Preprost primer pogojev Björlingovega problema je, da za krivuljo vzamemo kar $\alpha$, za vektorsko polje $\eta$ pa izberemo $\eta = N$.
Zanima nas kako izgleda parametrizacija minimalne ploskve, ki reši problem.

Vemo, da lahko splošno rešitev problema zapišemo kot:
$$ \phi (u, v) = Re \left( \alpha (z) - i \int_{z_0}^{z} \eta (w) \times \alpha' (w) dw \right) $$
Oglejmo si kako lahko razpišemo izraz pod integralom, saj imamo pogoj $\eta = N$.
$$ \eta \times \alpha' = N \times \alpha' (w) = - \alpha' (w) \times N = - | \alpha' (w) | (T \times N) = - | \alpha' (w) | B $$ 
Pri danih pogojih se rešitev Björlingovega problema glasi
$$ \phi (u, v) = Re \left( \alpha (z) + i \int_{z_0}^{z} B | \alpha' (w) | dw \right) $$

Pogoj $\eta = N$ ni vzet iz zraka, ampak je povezan z naslednjo definicijo krivulje.

\begin{definicija}
    Krivulja na ploskvi parametrizirana z naravno parametrizacijo je \emph{geodetka}, če je njen drugi odvod 
    povsod vzdolž krivulje vzporeden z enotsko normalo na ploskev. 
\end{definicija}

Ker je smer drugega odvoda ravno smer enotske normale $N$ po definiciji $N$, je v primeru, ko izberemo $\eta = N$,
izbrana krivulja geodetka na ploskvi, ki zadostuje pogojem Björlingovega problema.

Poglejmo si še primer, ko za vektorsko polje izberemo $\eta = - N(t)$

\begin{posledica}
    Naj bo $\phi (u, v)$ rešitev Björlingovega problema za krivuljo $\alpha (t)$ in vektorsko polje $N (t)$.
    Potem je rešitev Björlingovega problema za krivuljo $\alpha (t)$ in vektorsko polje $- N (t)$ parametrizirana z
    \begin{align*}
        \tilde{\phi} (u, v) &= Re \left[ \alpha(z) + i \int_{z_0}^{z} \eta (w) \times \alpha' (w) dw \right] \\
        &= \phi (u, - v)
    \end{align*}
\end{posledica}

\begin{dokaz}
    Naj bo $D$ domena kjer je rešitev $\phi (u, v)$ definirana. Prezrcalimo $D$ čez $u$-os in dobimo domeno $\tilde{D}$,
    na kateri definiramo $\tilde{\phi} (u, v) = \phi (u, - v)$.
    Ker ploskev $\tilde{\phi} (u, v)$ dobimo z zrcaljenjem, je tudi minimalna, njene normale pa definira zveza 
    $\tilde{\mathcal{N}} (u, v) = - \mathcal{N} (u, -v)$. Ploskev $\tilde{\phi} (u, v)$ je rešitev Björlingovega problema
    za vektorsko polje $- N(t)$.
    Ker je rešitev enolična, je 
    $$ \tilde{\phi} (u, v) = Re \left[ \alpha(z) + i \int_{z_0}^{z} \eta (w) \times \alpha' (w) dw \right] $$
\end{dokaz}

S posledico zgoraj smo rešitev definirano na $D$ razširili na $\tilde{D}$. Za $u \in I$ imajo $N(u)$, $\alpha' (u)$ in $z = u$
realne vrednosti, zato je $\tilde{\phi} (u, 0) = \alpha (u) = \phi (u, 0)$. Ker sta $D$ in $\tilde{D}$ odprti množici, ki vsebujeta
$I$, se ujemata na neki odprti množici. Po izreku identitete velja, da sta $\phi (u, v)$ in $\tilde{\phi}$ parametrizaciji iste 
ploskve kjer se $D$ in $\tilde{D}$ ujemata.


V dokazu naslednje leme privzemamo, da vektorsko polje $\eta$ zadošča pogojem Björlingovega problema.

\begin{lema}
    Naj bo $\phi (u, v)$ parametrizacija minimalne ploskve $M$ in naj bo $\phi (u, 0)$ krivulja v $xy$-ravnini.
    Če je ploskev $M$ pravokotna na $xy$-ravnino vzdolž $\phi (u, 0)$, veljajo naslednje zveze:
    \begin{align*}
        \phi^{1} (u, - v) &= \phi^{1} (u, v) \\
        \phi^{2} (u, - v) &= \phi^{2} (u, v) \\
        \phi^{3} (u, - v) &= - \phi^{3} (u, v) \\ 
    \end{align*}
\end{lema}

\begin{dokaz}
    Vzemimo za $\alpha(u)$ kar $\phi (u, 0)$ in naj bo vektorsko polje $\eta (u) = \mathcal{N} (u, 0)$ enotski normalni 
    vektor na ploskev $M$ vzdolž krivulje $\alpha$. Sedaj si lahko pomagamo z rešitvijo Björlingovega problema.
    $\alpha$ leži v $xy$-ravnini, zato jo lahko zapišemo kot $\alpha (u) = \left( \alpha^{1} (u), \alpha^{2} (u), 0 \right)$.
    Vemo tudi, da ploskev $M$ seka $xy$-ravnino pravokotno, zato je njeno vektorsko polje normal
    $\eta (u) = \mathcal{N} (u, 0) = \left( \eta^{1} (u), \eta^{2} (u), 0 \right)$.

    Del rešitve Björlingovega problema pod integralom lahko zato zapišemo:
    $$ \eta (u) \times \alpha' (u) = \left( 0, 0, \eta^1 (u) (\alpha^2)' (u) - \eta^2 (u) (\alpha^1)' (u) \right) $$

    Po izreku identitete velja, da za holomorfne razširitve $eta$ in $\alpha$ velja ista zveza, le da $u$ nadomestimo z $z$.
    Prva in druga koordinata $\eta (z) \times \alpha' (z)$ sta ničelni, zato nam rešitev Björlingovega problema da 
    $\phi^{1} (u, v) = Re \alpha^{1} (z)$ in $\phi^{2} (u, v) = Re \alpha^{2} (z)$. Z uporabo prejšnje posledice za $\alpha$
    in $- \eta$ pa veljaza zvezi $\phi^{1} (u, - v) = Re \alpha^{1} (z)$ in $\phi^{2} (u, - v) = Re \alpha^{2} (z)$.
    Tako smo dokazali prvi dve zvezi.

    Tretja koordinata $\alpha^{3} (z) = 0$, zato je $ \phi^{3} (u, v) = - Re \left( i \int \eta \times \alpha' dw \right) $.
    S ponovno uporabo prejšnje posledice za $\alpha$ in $- \eta$ dobimo zvezo 
    $ \phi^{3} (u, - v) = Re \left( i \int \eta \times \alpha' dw \right) = - \phi^{3} (u, v) $
\end{dokaz}


Podobno kot zgoraj, se da dokazati tudi naslednjo lemo.

\begin{lema}
    Naj bo $\phi (u, v)$ parametrizacija minimalne ploskve $M$ in naj bo $\phi (u, 0)$ v celoti vsebovana v $x$-osi.
    Potem veljajo naslednje zveze:
    \begin{align*}
        \phi^{1} (u, - v) &= \phi^{1} (u, v) \\
        \phi^{2} (u, - v) &= - \phi^{2} (u, v) \\
        \phi^{3} (u, - v) &= - \phi^{3} (u, v) \\ 
    \end{align*}
\end{lema}


Zgornji lemi nam dasta rezultat, s katerim lahko ploskev razširimo v prostoru $\mathbb{R}^3$.

\begin{izrek}{Schwarzov izrek simetrije ?}
    \begin{enumerate}
        \item Minimalna ploskev je simetrična glede na vsako premico vsebovano v ploskvi.
        \item Minimalna ploskev je simetrična glede na vsako ravnino, ki ploskev seka pravokotno.
    \end{enumerate}
\end{izrek}

Spomnimo se, da je za $\alpha$ in $\eta = N$ krivulja $\alpha$ geodetka na rešitev Björlingovega problema. 
Skupaj s Schwarzovim načelom simetrije nam da naslednji rezultat.

\begin{posledica}
    Če se $\eta$ ujema z ukrivljenostrjo $\alpha$, bo pripadajoča minimalna ploskev simetrična glede na $\alpha$.
\end{posledica}

\begin{primer}
    Scherkova prva ploskev, definirana z zvezo $$ z = c ln \left( \frac{cos(x/c)}{cos(y/c)} \right) $$
    je definirana na $ -\pi / 2 < x / c < \pi / 2, -\pi / 2 < y / c < \pi / 2$. Ko se bližamo ogljiščem 
    kvadrata, se približujemo nedefinirani vrednosti $0 / 0$.
    V tem primeru si pomagamo s Schwarzovim izrekom simetrije. Scherkovo ploskev razširimo na diagonalno 
    sosednji kvadrat kot na primer $ \pi / 2 < x / c < 3 \pi / 2, \pi / 2 < y / c < 3 \pi / 2$.

    V primeru Scherkove ploskve opazimo tudi, da je njen presek z $xy$-ravnino $y = \pm x$. To pa sta ravno 
    simetrijski osi ploskve.
\end{primer}


Poglejmo si še eno konstrukcijo minimalne ploskve s pomočjo Björlingovega problema.


Dano imamo krivuljo v $xz$-ravnini $\alpha (u) = \left( \beta (u), 0, \gamma (u) \right)$ z vektorskim 
poljem enotskih normal $N (u)$. Poiskati želimo parametrizacijo ploskve, ki reši Björlingov problem.

Tangentni vektor na krivuljo je $\alpha' (u) = \left( \beta' (u), 0, \gamma' (u) \right)$. Če ga zarotiramo
v pozitivni smeri in normiramo, dobimo enotsko normalo 
$$N = \frac{ \left( - \gamma', 0, \beta' \right) }{ \sqrt{(\beta')^2 + (\gamma')^2 }} $$.

Izračunamo vektorski produkt $N \times \alpha' = \left( 0, \sqrt{(\beta')^2 + (\gamma')^2}, 0 \right)$. 
Potem je realni del $- N \times \alpha'$ imaginarni del $N \times \alpha'$. Rešitev problema je torej:
$$ Re \left( \alpha - i \int N \times \alpha' \right) = \left( Re \beta, Im \int \sqrt{(\beta')^2 + (\gamma')^2}, Re \gamma \right)$$


Oglejmo si primer konstrukcije ploskve s pomočjo navedene izpeljave. V naslednjem primeru bomo poiskali
ploskev, ki vsebuje cikloido kot geodetko.

\begin{primer}{Catalanova ploskev}
    Naj bo $\alpha (u) = \left( 1 - cos u , 0, u - sin u \right)$ cikloida.
    Definirajmo kompleksno spremenljivko $z = u + i v$ in izračunamo
    \begin{align*}
        \phi^1 (u, v) &= Re (1 - cos z) = 1 - cos (u) cosh (v) \\
        \phi^2 (u, v) &= Re (z - sin z) = u - sin (u) cosh (v)
    \end{align*}
    Sedaj imamo prvo in tretjo koordinato rešitve.
    Nadalje izračunamo tangento na krivuljo $\alpha' = \left( sin z, 0, 1 - cos z \right)$. Potem je
    \begin{align*}
        sin^2 (w) + (1 - cos (w))^2 &= 2 - 2 cos(w) \\
        &= 4 \frac{1 - cos (w)}{2} \\
        &= 4 sin^2 (w/2)
    \end{align*}
    Druga koordinata je enaka
    $$ Im \int sqrt{sin^2 (w) + (1 - cos (w))^2} dw = Im \int 2 sin(w/2) dw = Im \left(- 4 cos(w/2) \right) = 4 sin(u/2) sinh(v/2)$$

    Parametrizacija Catalanove ploskve se glasi
    $$\phi (u, v) = \left( 1 - cos (u) cosh (v), 4 sin(u/2) sinh(v/2), u - sin (u) cosh (v) \right)$$
\end{primer}




dodaj 
\begin{itemize}
    \item  maš 8 ekvivalentnih def
    \item  Definicija - lokalna glede na okolice
    \item  def - s srednjo ukrivljenostjo
    \item  def - s enačbo minimalne ploskve (IZPELJI)
    \item  PRIMERI
    \begin{itemize}
        \item  Scherkova prva ploskev - maš izpeljavo v Oprah (str. 78 in 79)
        \item  če je revolucija  minimalna -> ravnina ali katenoida
        \item  če je ruled surface minimalen -> je del ravnine ali helikoida   
    \end{itemize}
    \item  Izotermne koo + def minimalne
    \item  Definicija glede na holomorfno
    \item  W-E reprezentacija + def glede na holomorfno
\end{itemize}

Zanima me, če pravilno razumem dokaz izreka 3.3 v diplomskem delu Katarine Gačnik. Za dokaz v desno predpostavimo, da minimalna ploskev obstaja in z $X$ označimo njeno parametrizacijo. Potem lahko za poljubno variacijo in za nek $\eta \in R$ definiramo $X_{\eta} = X + \eta \Phi n$. Zdaj definiramo preslikavo, ki vsaki $\eta$ priredi površino ploskve $X_{\eta}$: $\eta \rightarrow A[X_{\eta}]$. Ta preslikava bo imela minimum pri $\eta = 0$, ker smo predpostavili, da je $X$ minimalna. Iz tega potem sledi $$\frac{d}{d\eta} \Bigr|_{\eta = 0} A[X_{\eta}] = 0$$.



\end{document}