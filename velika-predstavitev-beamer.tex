\documentclass{beamer}

\usepackage[slovene]{babel}
\usepackage{amsfonts,amssymb}
\usepackage[utf8]{inputenc}
\usepackage{lmodern}
\usepackage[T1]{fontenc}
\usepackage{pdfpages}
\usepackage{graphicx}
\usepackage{subcaption}
\usepackage{multicol}

\usetheme{Boadilla}

\def\qed{$\hfill\Box$}   % konec dokaza
\newtheorem{izrek}{Izrek}
\newtheorem{trditev}{Trditev}
\newtheorem{posledica}{Posledica}
\newtheorem{lema}{Lema}
\newtheorem{definicija}{Definicija}
\newcommand{\R}{\mathbb R}

\title{Minimalne ploskve in Björlingov problem}
\author{Lucija Fekonja \\
Mentor: Doc. dr. Uroš Kuzman}
\institute{Fakulteta za matematiko in fiziko \\
Oddelek za matematiko}
\date{30. maj 2023}

\begin{document}

    \begin{frame}
        \titlepage
    \end{frame}

    % ------------------------------------------------------------------------------------------------------------------------

    \begin{frame}{Minimalna ploskev}
        \begin{definicija}
            Ploskev $M \subset \mathbb{R}^3$ je \emph{minimalna ploskev}, če in samo če obstaja okolica vsake točke $p \in M$, omejena s 
            sklenjeno krivuljo, ki ima najmanjšo površino med vsemi ploskvami z isto omejitvijo.
        \end{definicija}

        \pause

        \begin{definicija}
            Ploskev se imenuje \emph{minimalna ploskev}, če je njena srednja ukrivljenost enaka nič.
        \end{definicija}
    \end{frame}

    % ------------------------------------------------------------------------------------------------------------------------

    \begin{frame}{Catalanova minimalna ploskev}
        $$\phi (u, v) = \left( 1 - cos (u) cosh (v), 4 sin(u/2) sinh(v/2), u - sin (u) cosh (v) \right)$$

        \begin{figure}[h]
            \includegraphics[height = 4cm]{Catalanova-minimalna-ploskev.eps}
            \caption{Catalanova minimalna ploskev}
        \end{figure}
    \end{frame}
    
    % ------------------------------------------------------------------------------------------------------------------------

    \begin{frame}{Björlingov problem}
        Naj velja:
        \begin{align*}
            \alpha &: I \mapsto \R^3 \text{realno analitična krivulja} \\
            \eta &: I \mapsto \R^3 \text{realno analitično vektorsko polje} \\
            \left\lvert \eta (t) \right\rvert &= 1 \\
            \eta (t) \cdot \alpha' (t) &= 0
        \end{align*}

        Najdi parametrizacijo minimalne ploskve $\phi (u, v)$, za katero velja:
        \begin{enumerate}
            \item Ploskev naj vsebuje krivuljo $\alpha$ pri $v = 0$. To pomeni, $\forall u \in I. \alpha (u) = \phi (u, 0)$.
            \item Normale na ploskev se naj vzdolž celotne krivulje $\alpha$ ujemajo z vektorji vektorskega polja $\eta$: 
            $\forall u \in I. \eta (u) = N(u, 0)$.
        \end{enumerate}
    \end{frame}

% ------------------------------------------------------------------------------------------------------------------------

    \begin{frame}{Rešitev Björlingovega problema}
        $$ \phi (u, v) = Re \left( \alpha (z) - i \int_{z_0}^{z} \eta (w) \times \alpha' (w) dw \right) $$
    \end{frame}

    % ------------------------------------------------------------------------------------------------------------------------

    \begin{frame}{Princip identičnosti}
        \begin{izrek}
            Naj bosta $f$ in $g$ holomorfni funkciji na povezanem odprtem območju $D \subseteq \mathbb{C}$ (ali $\mathbb{R}$).
            Če ima $S \subseteq D$ stekališče, potem je $f = g$ na $D$.
        \end{izrek}
    \end{frame}

    % ------------------------------------------------------------------------------------------------------------------------

    \begin{frame}{Izotermna parametrizacija}
        \pause
        \begin{lema}
            $\phi$ ima izotermne koordinate natanko tedaj, ko velja $(\phi')^2 = 0$.
        \end{lema}

        \pause
        \begin{lema}
            Ploskev $M$ z izotermno parametrizacijo $\phi = \left( \phi^1, \phi^2, \phi^3 \right)$
            je minimalna natanko tedaj, ko so $\phi^1, \phi^2$ in $\phi^3$ harmonične.
        \end{lema}
    \end{frame}

    % ------------------------------------------------------------------------------------------------------------------------

    \begin{frame}{Dokaz rešitve Björlingovega problema}
        $$ \phi (u, v) = Re \left( \alpha (z) - i \int_{z_0}^{z} \eta (w) \times \alpha' (w) dw \right) $$
    \end{frame}
        
    % ------------------------------------------------------------------------------------------------------------------------

    \begin{frame}{Konstrukcija minimalne ploskve}
        Krivulja: $\alpha (u) = \left( \beta (u), 0, \gamma (u) \right)$

        Normale: $N = \frac{ \left( - \gamma', 0, \beta' \right) }{ \sqrt{(\beta')^2 + (\gamma')^2 }} $

        Izračunamo vektorski produkt $N \times \alpha' = \left( 0, \sqrt{(\beta')^2 + (\gamma')^2}, 0 \right)$

        Rešitev problema je torej:
        $$ Re \left( \alpha - i \int N \times \alpha' \right) = \left( Re \beta, Im \int \sqrt{(\beta')^2 + (\gamma')^2}, Re \gamma \right)$$
    \end{frame}

    % ------------------------------------------------------------------------------------------------------------------------

    \begin{frame}{Catalanova minimalna ploskev}
        $$\phi (u, v) = \left( 1 - cos (u) cosh (v), 4 sin(u/2) sinh(v/2), u - sin (u) cosh (v) \right)$$

        \begin{figure}[h]
            \includegraphics[height = 4cm]{Catalanova-minimalna-ploskev.eps}
            \caption{Catalanova minimalna ploskev}
        \end{figure}
    \end{frame}
    

\end{document}