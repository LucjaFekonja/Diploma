\documentclass{article}
\usepackage{amsmath}
\usepackage{amsfonts}
\usepackage{graphicx}
\usepackage{xcolor}

\author{Lucija Fekonja \\ Mentor: doc. dr. Uroš Kuzman}
\title{Minimalne ploskve in Björlingov problem}

\newcommand{\R}{\mathbb{R}}
\newcommand{\N}{\mathbb N}
\newcommand{\Z}{\mathbb Z}
\newcommand{\C}{\mathbb C}
\newcommand{\Q}{\mathbb Q}

\newtheorem{definicija}{Definicija}
\newtheorem{lema}{Lema}
\newtheorem{dokaz}{Dokaz}
\newtheorem{trditev}{Trditev}
\newtheorem{izrek}{Izrek}
\newtheorem{posledica}{Posledica}
\newtheorem{primer}{Primer}


% Še neke ostale stvari ...


\begin{document}

    \maketitle

    \section{Definicija minimalne ploskve}

        	Minimalno ploskev lahko definiramo na osem različnih, a med sabo ekvavilentnih načinov.
        	Lokalno definicijo minimalne ploskve lahko zapišemo na sledeč način.
        	
        	\begin{definicija}
        	    Ploskev $M \subset \mathbb{R}^3$ je \emph{minimalna ploskev}, če in samo če obstaja okolica vsake točke $p \in M$, omejena s 
        	    sklenjeno krivuljo, ki ima najmanjšo površino med vsemi ploskvami z isto omejitvijo.
        	\end{definicija}
        	
        	
        	Intuitivno je minimalna ploskev takšna ploskev, ki ima najmanjšo površino pri nekih robnih pogojih. 
        	Pogosteje jih definiramo z uporabo srednje ukrivljenosti, saj jih na ta način podamo globalno.
        	Zavoljo te definicije spoznajmo nekaj osnovnih pojmov vezanih na ploskve v splošnem.
    	
        	\begin{definicija}
        	    \emph{Enotsko normalo $N$} regularne ploskve s parametrizacijo $\phi(u, v)$ definiramo kot vektor $N = \frac{\phi_u \times \phi_v}{\left\lvert \phi_u \times \phi_v \right\rvert }$.
        	\end{definicija}
    	
        	\begin{definicija}
        	    \emph{Normalna ukrivljenost $\kappa_n$} je komponenta ukrivljenosti $\kappa$ ploskovne krivulje $\gamma$ v smeri normale
        	    $$ \kappa_n = \frac{dT}{dS} \cdot N. $$
        	\end{definicija}
    	
        	\begin{definicija}
        	    Naj bo $p$ točka na ploskvi. Poglejmo vse krivulje $\gamma_i$ na ploskvi, ki gredo skozi točko $p$. Naj bo $\kappa_1$ maksimalna izmed normalnih ukrivljenosti
        	    teh krivulj v točki $p$, $\kappa_2$ pa minimalna. 
        	    \emph{Srednja ukrivljenost H} je definirana kot $H = \frac{ \kappa_1 + \kappa_2 }{2}$.
        	\end{definicija}
        	
        	Sedaj lahko globalno definiramo minimalno ploskev.
        	
        	\begin{definicija}
        	    Ploskev se imenuje \emph{minimalna ploskev}, če je njena srednja ukrivljenost enaka nič.
        	\end{definicija}


    \section{Björlingov problem}

        Danes bi na predstavitvi rada opisala drugi del svojega diplomskega dela, torej Björlingov problem.

        Marsikdo je verjetno že slišal za Plateaujev problem minimalne ploskve, ki zahteva, da najdemo minimalno
        ploskev s predpisanim robom. Pogoj, da je krivulja rob ploskve je precej omejujoč. Nekoliko splošnejši problem bi bil,
        da zahtevamo, da ploskev vsebuje neko krivuljo. Tako pridem do Björlingovega problema.

        Recimo, da je $\alpha (t) : I \mapsto \R^3$ realno analitična krivulja in $\eta : I \mapsto \R^3$ 
        realno analitično vektorsko polje, da velja $ \left\lvert \eta \right\rvert = 1$ in $\eta (t) \cdot \alpha' (t) = 0$ 
        za vse točke $t$ iz intervala $I$. Drugi pogoj pove, da je $\eta$ pravokoten na vse tangentne vektorje krivulje $\alpha$.

        Björlingov problem zahteva, da najdemo takšno parametrizacijo $\phi (u, v)$ minimalne ploskve $M$, za katero velja:
        \begin{enumerate}
            \item Ploskev $M$ naj vsebuje krivuljo $\alpha$ pri $v = 0$. Tj. $\forall u \in I. \alpha (u) = \phi (u, 0)$
            \item Normale na ploskev $M$ se naj vzdolž celotne krivulje $\alpha$ ujemajo z vektorji vektorskega polja $\eta$ 
            Tj. $\forall u \in I. \eta (u) = \mathcal{N} (u, 0)$
        \end{enumerate}

        Izkaže se, da je rešitev Björlingovega problema ena sama in jo lahko zapišemo eksplicitno:
        $$ \phi (u, v) = Re \left( \alpha (z) - i \int_{z_0}^{z} \eta (w) \times \alpha' (w) dw \right) $$

        Da je dana ploskev res rešitev zadanega problema je leta 1890 dokazal Schwarz v dveh korakih. Dokazal je enoličnost
        in obstoj rešitve. Namesto da bi danes na predstavitvi dokazovala kar je dokazal Schwarz, bi raje izpeljala nekaj posledic 
        in konstruirala kakšno minimialno ploskev s pomočjo rešitve.


    \section{Dodatki za dokaz rešitve}

        \subsection{Princip identičnosti}

            Za dokaz rešitve problema, pa tudi za dokaz posledic pa potrebujemo nekaj znanja iz kompleksne analize, 
            specifično je dobro poznati princip identičnosti, ki pravi naslednje:

            \begin{izrek}{(Princip identičnossti)}
                Naj bosta $f$ in $g$ holomorfni funkciji na povezanem odprtem območju $D \subseteq \mathbb{C}$ (ali $\mathbb{R}$).
                Če ima $S \subseteq D$ stekališče, potem je $f = g$ na $D$.
            \end{izrek}


        \subsection{Izotermna parametrizacija}

            Naj bo zdaj $\gamma$ krivulja z enotsko hitrostjo, to pomeni, da je $\left\lvert \gamma' \right\rvert = 1$ povsod na njenem definicijskem območju.
            Potem je $\gamma' \cdot \gamma' = 1$.
            Navedeno lahko razpišemo:
            
            \begin{align*}
                1 &= \gamma' \cdot \gamma' \\
                &= (u' \phi_u + v' \phi_v) \cdot (u' \phi_u + v' \phi_v)\\
                &= (\phi_u \cdot \phi_u) u'^2 + (\phi_u \cdot \phi_v + \phi_v \cdot \phi_u) u' v' + (\phi_v \cdot \phi_v) v'^2  \\
                &= I(\phi_u) u'^2 + 2 I(\phi_u, \phi_v) u' v' + I(\phi_v) v'^2\\
                &= E u'^2 + 2F u' v' + G v'^2\\
            \end{align*}

            Označili smo:
            \begin{align*}
                E &= I(\phi_u) = \phi_u \cdot \phi_u = \left\lvert \phi_u \right\rvert^2 \\
                F &= I(\phi_u, \phi_v) = \phi_u \cdot \phi_v \\
                G &= I(\phi_v) = \phi_v \cdot \phi_v = \left\lvert \phi_v \right\rvert^2
            \end{align*}

            Koeficienti $E, F \text{ in } G$ se imenujejo \emph{koeficienti prve fundamentalne forme $I$}.


            
            \begin{definicija}
                Parametrizacije ploskve $phi$ je \emph{izotermna}, če velja $$ E = \phi_u \cdot \phi_u = \phi_v 
                \cdot \phi_v = G \qquad \text{ in } \qquad F = \phi_u \cdot \phi_v = 0 $$ 
            \end{definicija}

            Omenimo tukaj, da ima vsaka minimalna ploskev izotermne koordinate. Dokaz opuščam.

            Prav tako pa potrebujemo naslednjo lemo.

            \begin{lema}
                $\phi$ ima izotermne koordinate natanko tedaj, ko velja $(\phi')^2 = 0$.
            \end{lema}

            \begin{dokaz}
                Naj bo $\phi$ izotermna parametrizacija. Potem je $\left( \phi_z \right)^2 = \frac{1}{4} \left( E - G - 2i F \right) 
                = \frac{1}{4} \left( E - E - 2i \cdot 0 \right) = 0$
            
                Obratno, naj bo $\left( \phi_z \right)^2 = 0$. Potem je $\frac{1}{4} \left( E - G - 2i F \right) = \frac{E}{4} - \frac{G}{4} - i \frac{F}{2} = 0 + i \cdot 0$.
                Dobimo sistem dveh enačb $\frac{E}{4} - \frac{G}{4} = 0$ in $\frac{F}{2} = 0$ iz česar seveda sledi $E = G$ in $F = 0$. Tako je naša lema dokazana.
            \end{dokaz}


    
    \section{Dokaz rešitve Björlingovega problema}

        Rešitev bomo dokazali v dveh korakih. Najprej bomo privzeli, da je $\phi$ parametrizacija rešitve in pokazali,
        da je takšna kot v izreku. Iz tega bo sledilo, da je vsaka rešitev Björlingovega problema takšne oblike 
        kot smo jo navedli in je torej enolična. Nato bomo še pokazali, da rešitvena parametrizacija zadošča 
        obema pogojema Björlingovega problema, kar pa pomeni, da res obstaja.


        \begin{dokaz}
            Recimo, da je $\phi$ rešitev Björlingovega problema.
            $\phi$ torej parametrizira minimalno ploskev $M$.
            Ker za vsako minimalno ploskev obstaja izotermna parametrizacija, predpostavimo, da je $\phi$ izotermna.
            Po prejšnji lemi je $\phi$ harmonična: $\Delta \phi = 0$.
        
            Naj bo $\varphi^{j}$ harmonična konjungiranka $\phi^{j}$ tako, da je $\phi^{i} + i \varphi^{i}$ holomorfna.
            Definirajmo holomorfno funkcijo
            \begin{align*}
                & \beta (z) : D \mapsto \C^3 \\
                & \beta (z) = \phi (z) + i \varphi (z) = \left[
                \begin{matrix}
                    \phi^{1} \\
                    \phi^{2} \\
                    \phi^{3} \\
                \end{matrix} \right]
                + i \left[
                \begin{matrix}
                    \varphi^{1} \\
                    \varphi^{2} \\
                    \varphi^{3} \\
                \end{matrix} \right].
            \end{align*}
        
            $\phi$ in $\varphi$ sta funkciji v realnih spremenljivkah $u$ in $v$, zato lahko tudi $\beta$ odvajamo po $u$ in $v$.
            Odvod $\beta$ po $u$ je $\frac{\delta}{\delta u} \beta(z) = \frac{\delta}{\delta u} (\phi + i \varphi) = \phi_u + i \varphi_u$.
            Ker je $\beta$ holomorfna, ustreza Cauchy-Riemannovim pogojem, zato lahko odvod zapišemo kot $\frac{\delta}{\delta u} \beta(z) = \phi_u - i \phi_v$.
        
            Predpostavili smo, da je $phi$ izotermna. Po definiciji izotermnosti velja, da sta si vektorja hitrosti $\phi_u$ in $\phi_v$
            pravokotna. Normala $N$ na ploskev pa je pravokotna na oba vektorja hitrosti, zato je $\phi_v = N \times \phi_u$.
        
            Sledi $\beta'(z) = \phi_u(z) - i \cdot \left( N(z) \times \phi_u(z) \right)$.
            Če se sedaj omejimo le na realne vrednosti $z$, dobimo 
            $$\beta'(u) = \alpha'(u) - i \cdot \left( \mathcal{N} (u) \times \alpha'(u) \right)$$
            za $u \in I$, kjer je $I$ kot v pogojih Björlingovega problema. Enakost velja, ker je $\phi(u, 0) = \alpha(u)$ in $N(u, 0) = \mathcal{N} (u)$.
        
            Izraz lahko integriramo po realni spremenljivki $u$ in dobimo:
            $$\beta(u) = \alpha(u) - i \cdot \int_{u_0}^{u} \mathcal{N} (t) \times \alpha'(t) dt$$.
        
            Recimo sedaj, da je $\gamma(z) = \alpha(z) - i \cdot \int_{u_0}^{z} \mathcal{N} (w) \times \alpha'(w) dw$ holomorfna krivlulja.
            Opazimo, da je ta vsak $u \in I, \beta(u) = \gamma(u + i \cdot 0)$. Zato po izreku identitete velja $\beta(z) = \gamma(z)$ za vsak $z \in D$.

            Realni del funkcije $\beta$ je ravno parametrizacija $\phi = Re(\phi + i \cdot \varphi)$. Torej je 
            $$ \phi(u, v) = Re(\beta(z)) = Re(\alpha(z) - i \cdot \int_{u_0}^{z} \mathcal{N} (w) \times \alpha'(w) dw)$$,
            kar je pa ravno rešitev problema.
        
            Dokazali smo, da če $\phi$ parametrizira ploskev in zadošča pogojem problema, potem je ravno takšne oblike kot je 
            navedeno v rešitvi. To pomeni, da je rešitev enolična.
        
            Za dokaz obstoja, naj bo $\beta(z)$ holomorna funkcija definirana kot 
            \begin{align*}
                \beta (z) &= \phi(z) + i \cdot \varphi(z) \\
                &= \alpha(z) - i \cdot \int_{u_0}^{z} \mathcal{N} (w) \times \alpha'(w) dw
            \end{align*}
        
            Naj bo $u \in I$. Vemo, da sta $\alpha'(u)$ in $\mathcal{N} (u)$ realni. Omejimo sedaj funkcijo $\beta$ na $u$. 
            \begin{align*}
                &\beta(u) = \alpha(u) - i \cdot \int_{u_0}^{u} \mathcal{N} (t) \times \alpha'(t) dt \\
                \Rightarrow \qquad &\beta'(u) = \alpha'(u) - i \cdot \left( \mathcal{N} (u) \times \alpha'(u) \right)
            \end{align*}
        
            Vemo, da je $\mathcal{N} \times \alpha' \bot \alpha'$, zato je $\left( \mathcal{N} \times \alpha' \right) \cdot \alpha'$.
        
            Prav tako vemo, da je $\mathcal{N} \bot \alpha'$, zato velja $|\mathcal{N} \times \alpha'| = |\mathcal{N}| \cdot |\alpha'| = |\alpha'|$.
        
            Izračunajmo:
            \begin{align*}
                \beta'(u)^2 &= \left( \alpha'(u) - i \cdot \left( \mathcal{N} (u) \times \alpha'(u) \right) \right)^2 
                &= \alpha'(u) \cdot \alpha'(u) - 2i \alpha'(u) \cdot (\mathcal{N} \times \alpha'(u)) - (\mathcal{N} \times \alpha'(u)) \cdot (\mathcal{N} \times \alpha'(u))
                &= |\alpha'(u)|^2 - 0 - |\mathcal{N} \times \alpha'(u)|^2
                &= |\alpha'(u)|^2 - |\alpha'(u)|^2
                &= 0
            \end{align*}
        
            Po izreku identitete iz $\beta'(u)^2 = 0$ za $\forall u \in I$ sledi $\beta'(u)^2 = 0$ za $\forall u \in D$.
            To pa je ravno pogoj za obstoj izotermnih koordinat po prejšnji lemi.
            Torej realni del funkcije $\beta$ parametrizira minimalno ploskev $M$ v izotermnih koordinatah.
            Tako smo dokazali obstoj izotermne parametrizacije ploskve, ki reši Björlingov problem.
        \end{dokaz} 


    
    \section{Geodetka in primer Björlingovega problema}

        Predpostavimo sedaj, da imamo naravno parametrizacijo krivulje $\alpha(s)$, ki ima enotski tangentni vektor $T = \alpha'(s)$.
        Naj bo $N = \frac{T'}{| T' |} = \frac{\alpha''}{| \alpha |}$ njena enotska normala in $B = T \times N$ njena stranska normala.
        Preprost primer pogojev Björlingovega problema je, da za krivuljo vzamemo kar $\alpha$, za vektorsko polje $\eta$ pa izberemo $\eta = N$.
        Zanima nas kako izgleda parametrizacija minimalne ploskve, ki reši problem.

        Vemo, da lahko splošno rešitev problema zapišemo kot:
        $$ \phi (u, v) = Re \left( \alpha (z) - i \int_{z_0}^{z} \eta (w) \times \alpha' (w) dw \right) $$
        Oglejmo si kako lahko razpišemo izraz pod integralom, saj imamo pogoj $\eta = N$.
        $$ \eta \times \alpha' = N \times \alpha' (w) = - \alpha' (w) \times N = - | \alpha' (w) | (T \times N) = - | \alpha' (w) | B $$ 
        Pri danih pogojih se rešitev Björlingovega problema glasi
        $$ \phi (u, v) = Re \left( \alpha (z) + i \int_{z_0}^{z} B | \alpha' (w) | dw \right) $$

        Pogoj $\eta = N$ ni vzet iz zraka, ampak je povezan z naslednjo definicijo krivulje.

        \begin{definicija}
            Krivulja na ploskvi parametrizirana z naravno parametrizacijo je \emph{geodetka}, če je njen drugi odvod 
            povsod vzdolž krivulje vzporeden z enotsko normalo na ploskev. 
        \end{definicija}

        Ker je smer drugega odvoda ravno smer enotske normale $N$ po definiciji $N$, je v primeru, ko izberemo $\eta = N$,
        izbrana krivulja geodetka na ploskvi, ki zadostuje pogojem Björlingovega problema.

        Poglejmo si še primer, ko za vektorsko polje izberemo $\eta = - N(t)$

        \begin{posledica}
            Naj bo $\phi (u, v)$ rešitev Björlingovega problema za krivuljo $\alpha (t)$ in vektorsko polje $N (t)$.
            Potem je rešitev Björlingovega problema za krivuljo $\alpha (t)$ in vektorsko polje $- N (t)$ parametrizirana z
            \begin{align*}
                \tilde{\phi} (u, v) &= Re \left[ \alpha(z) + i \int_{z_0}^{z} \eta (w) \times \alpha' (w) dw \right] \\
                &= \phi (u, - v)
            \end{align*}
        \end{posledica}

        \begin{dokaz}
            Naj bo $D$ domena, kjer je rešitev $\phi (u, v)$ definirana. Prezrcalimo $D$ čez $u$-os in dobimo domeno $\tilde{D}$,
            na kateri definiramo $\tilde{\phi} (u, v) = \phi (u, - v)$.
            Ker ploskev $\tilde{\phi} (u, v)$ dobimo z zrcaljenjem, je tudi minimalna, njene normale pa definira zveza 
            $\tilde{\mathcal{N}} (u, v) = - \mathcal{N} (u, -v)$. Ploskev $\tilde{\phi} (u, v)$ je rešitev Björlingovega problema
            za vektorsko polje $- N(t)$.
            Ker je rešitev enolična, je 
            $$ \tilde{\phi} (u, v) = Re \left[ \alpha(z) + i \int_{z_0}^{z} \eta (w) \times \alpha' (w) dw \right] $$
        \end{dokaz}

        S posledico zgoraj smo rešitev definirano na $D$ razširili na $\tilde{D}$. Za $u \in I$ imajo $N(u)$, $\alpha' (u)$ in $z = u$
        realne vrednosti, zato je $\tilde{\phi} (u, 0) = \alpha (u) = \phi (u, 0)$. Ker sta $D$ in $\tilde{D}$ odprti množici, ki vsebujeta
        $I$, se ujemata na neki odprti množici. Po izreku identitete velja, da sta $\phi (u, v)$ in $\tilde{\phi}$ parametrizaciji iste 
        ploskve kjer se $D$ in $\tilde{D}$ ujemata.


    \section{Schwarzov izrek}

        V dokazu naslednje leme privzemamo, da vektorsko polje $\eta$ zadošča pogojem Björlingovega problema.

        \begin{lema}
            Naj bo $\phi (u, v)$ parametrizacija minimalne ploskve $M$ in naj bo $\phi (u, 0)$ krivulja v $xy$-ravnini.
            Če je ploskev $M$ pravokotna na $xy$-ravnino vzdolž $\phi (u, 0)$, veljajo naslednje zveze:
            \begin{align*}
                \phi^{1} (u, - v) &= \phi^{1} (u, v) \\
                \phi^{2} (u, - v) &= \phi^{2} (u, v) \\
                \phi^{3} (u, - v) &= - \phi^{3} (u, v) \\ 
            \end{align*}
        \end{lema}

        \begin{dokaz}
            Vzemimo za $\alpha(u)$ kar $\phi (u, 0)$ in naj bo vektorsko polje $\eta (u) = \mathcal{N} (u, 0)$ enotski normalni 
            vektor na ploskev $M$ vzdolž krivulje $\alpha$. Sedaj si lahko pomagamo z rešitvijo Björlingovega problema.
            $\alpha$ leži v $xy$-ravnini, zato jo lahko zapišemo kot $\alpha (u) = \left( \alpha^{1} (u), \alpha^{2} (u), 0 \right)$.
            Vemo tudi, da ploskev $M$ seka $xy$-ravnino pravokotno, zato je njeno vektorsko polje normal
            $\eta (u) = \mathcal{N} (u, 0) = \left( \eta^{1} (u), \eta^{2} (u), 0 \right)$.
        
            Del rešitve Björlingovega problema pod integralom lahko zato zapišemo:
            $$ \eta (u) \times \alpha' (u) = \left( 0, 0, \eta^1 (u) (\alpha^2)' (u) - \eta^2 (u) (\alpha^1)' (u) \right) $$
        
            Po izreku identitete velja, da za holomorfne razširitve $\eta$ in $\alpha$ velja ista zveza, le da $u$ nadomestimo z $z$.
            Prva in druga koordinata $\eta (z) \times \alpha' (z)$ sta ničelni, zato nam rešitev Björlingovega problema da 
            $\phi^{1} (u, v) = Re \alpha^{1} (z)$ in $\phi^{2} (u, v) = Re \alpha^{2} (z)$. Z uporabo prejšnje posledice za $\alpha$
            in $- \eta$ pa veljaza zvezi $\phi^{1} (u, - v) = Re \alpha^{1} (z)$ in $\phi^{2} (u, - v) = Re \alpha^{2} (z)$.
            Tako smo dokazali prvi dve zvezi.
        
            Tretja koordinata $\alpha^{3} (z) = 0$, zato je $ \phi^{3} (u, v) = - Re \left( i \int \eta \times \alpha' dw \right) $.
            S ponovno uporabo prejšnje posledice za $\alpha$ in $- \eta$ dobimo zvezo 
            $ \phi^{3} (u, - v) = Re \left( i \int \eta \times \alpha' dw \right) = - \phi^{3} (u, v) $
        \end{dokaz}


        Podobno kot zgoraj, se da dokazati tudi naslednjo lemo.

        \begin{lema}
            Naj bo $\phi (u, v)$ parametrizacija minimalne ploskve $M$ in naj bo $\phi (u, 0)$ v celoti vsebovana v $x$-osi.
            Potem veljajo naslednje zveze:
            \begin{align*}
                \phi^{1} (u, - v) &= \phi^{1} (u, v) \\
                \phi^{2} (u, - v) &= - \phi^{2} (u, v) \\
                \phi^{3} (u, - v) &= - \phi^{3} (u, v) \\ 
            \end{align*}
        \end{lema}


        Zgornji lemi nam dasta rezultat, s katerim lahko ploskev razširimo v prostoru $\mathbb{R}^3$.

        \begin{izrek}{Schwarzov izrek}
            \begin{enumerate}
                \item Minimalna ploskev je simetrična glede na vsako premico vsebovano v ploskvi.
                \item Minimalna ploskev je simetrična glede na vsako ravnino, ki ploskev seka pravokotno.
            \end{enumerate}
        \end{izrek}


        \begin{primer}
            Scherkova prva ploskev, definirana z zvezo $$ z = c ln \left( \frac{cos(x/c)}{cos(y/c)} \right) $$
            je definirana na $ -\pi / 2 < x / c < \pi / 2, -\pi / 2 < y / c < \pi / 2$. Ko se bližamo ogljiščem 
            kvadrata, se približujemo nedefinirani vrednosti $0 / 0$.
            V tem primeru si pomagamo s Schwarzovim izrekom simetrije. Scherkovo ploskev razširimo na diagonalno 
            sosednji kvadrat kot na primer $ \pi / 2 < x / c < 3 \pi / 2, \pi / 2 < y / c < 3 \pi / 2$.
        
            V primeru Scherkove ploskve opazimo tudi, da je njen presek z $xy$-ravnino $y = \pm x$. To pa sta ravno 
            simetrijski osi ploskve.
        \end{primer}


    \section{Konstrukcija minimalne ploskve}

        Poglejmo si še eno konstrukcijo minimalne ploskve s pomočjo Björlingovega problema.


        Dano imamo krivuljo v $xz$-ravnini $\alpha (u) = \left( \beta (u), 0, \gamma (u) \right)$ z vektorskim 
        poljem enotskih normal $N (u)$. Poiskati želimo parametrizacijo ploskve, ki reši Björlingov problem.

        Tangentni vektor na krivuljo je $\alpha' (u) = \left( \beta' (u), 0, \gamma' (u) \right)$. Če ga zarotiramo
        v pozitivni smeri in normiramo, dobimo enotsko normalo 
        $$N = \frac{ \left( - \gamma', 0, \beta' \right) }{ \sqrt{(\beta')^2 + (\gamma')^2 }} $$.

        Izračunamo vektorski produkt $N \times \alpha' = \left( 0, \sqrt{(\beta')^2 + (\gamma')^2}, 0 \right)$. 
        Potem je realni del $- i N \times \alpha'$ imaginarni del $N \times \alpha'$. Rešitev problema je torej:
        $$ Re \left( \alpha - i \int N \times \alpha' \right) = \left( Re \beta, Im \int \sqrt{(\beta')^2 + (\gamma')^2}, Re \gamma \right)$$


        Oglejmo si primer konstrukcije ploskve s pomočjo navedene izpeljave. V naslednjem primeru bomo poiskali
        ploskev, ki vsebuje cikloido kot geodetko.

        \begin{primer}{(Catalanova ploskev)}
            Naj bo $\alpha (u) = \left( 1 - cos u , 0, u - sin u \right)$ cikloida.
            Definirajmo kompleksno spremenljivko $z = u + i v$ in izračunamo
            \begin{align*}
                \phi^1 (u, v) &= Re (1 - cos z) = 1 - cos (u) cosh (v) \\
                \phi^3 (u, v) &= Re (z - sin z) = u - sin (u) cosh (v)
            \end{align*}
            Sedaj imamo prvo in tretjo koordinato rešitve.
            Nadalje izračunamo tangento na krivuljo $\alpha' = \left( sin z, 0, 1 - cos z \right)$. Potem je
            \begin{align*}
                sin^2 (w) + (1 - cos (w))^2 &= 2 - 2 cos(w) \\
                &= 4 \frac{1 - cos (w)}{2} \\
                &= 4 sin^2 (w/2)
            \end{align*}
            Druga koordinata je enaka
            $$ Im \int \sqrt{sin^2 (w) + (1 - cos (w))^2} dw = Im \int 2 sin(w/2) dw = Im \left(- 4 cos(w/2) \right) = 4 sin(u/2) sinh(v/2)$$
        
            Parametrizacija Catalanove ploskve se glasi
            $$\phi (u, v) = \left( 1 - cos (u) cosh (v), 4 sin(u/2) sinh(v/2), u - sin (u) cosh (v) \right)$$
        \end{primer}


\end{document}